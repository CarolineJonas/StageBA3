% !TeX spellcheck = en_US
\documentclass[a4paper,12pt]{article}
\def\xmu{x^\mu}
\def\vecx{\vec{x}}
\def\CP{\mathcal{P}}
\def\CL{\mathcal{L}}
\def\pt{P_\tau}
\def\vecpt{\vec{\pt}}
\usepackage[utf8]{inputenc}
\usepackage{amsmath}
\usepackage{amssymb}
\usepackage[a4paper]{geometry}
\geometry{hscale=0.80,vscale=0.80,centering}
\title{Calcul des crochets de Poissons pour les modes normaux.}


\begin{document}
\maketitle

Nous allons calculer les crochets de Poissons parmi les variables indépendantes et les modes dépendants.\\
Lors du calcul des contraintes sur les $a_n^-$, on a montré que:
\begin{equation}
\left\lbrace
\begin{aligned}
&x^i= q_0^i + \sqrt{2}a_0^i \tau + \sqrt{2}i\sum_{n=-\infty;n\neq 0}^{+\infty}\frac{a_n^i}{n}\cos(n\sigma)e^{-in\tau}\\
&\pt^i=\sum_{k=-\infty}^{+\infty}\frac{a_k^i}{\sqrt{2}\pi}\cos(k\sigma)e^{-ik\tau}
\end{aligned}
\right.
\end{equation}
Donc si on veut isoler les $a_n^i$ en fonction de $x^i$ et $\pt^i$:
\begin{flalign*}
\int_0^\pi \pt^i \cos(m\sigma)d\sigma&=\sum_{k=-\infty}^{+\infty}\frac{a_k^i}{\sqrt{2}\pi}e^{-ik\tau}\int_0^\pi\cos(k\sigma)cos(m\sigma)d\sigma=\sum_{k=-\infty}^{+\infty}\frac{a_k^i}{\sqrt{2}\pi}e^{-ik\tau}\frac{\pi}{2}\left( \delta_{k,m}+\delta_{k,-m}\right) \\
&=\frac{1}{2\sqrt{2}}\left(a_m^ie^{-im\tau}+a_{-m}^ie^{im\tau}\right) 
\end{flalign*}
\begin{flalign*}
\int_0^\pi x^i \cos(m\sigma)d\sigma &=  (q_0^i + \sqrt{2}a_0^i \tau)\int_0^\pi\cos(m\sigma)d\sigma + \sqrt{2}i\sum_{n=-\infty;n\neq 0}^{+\infty}\frac{a_n^i}{n}e^{-in\tau}\int_0^\pi\cos(n\sigma)\cos(m\sigma)d\sigma\\
&=\pi\delta_{m,0}(q_0^i + \sqrt{2}a_0^i \tau) + \epsilon_{m,0}\frac{\pi\sqrt{2}i}{2}\left( \frac{a_m^i}{m}e^{-im\tau} + \frac{a_{-m}^i}{-m}e^{im\tau})\right) 
\end{flalign*}
Donc $\forall m\neq 0$:
$$a^i_m=e^{im\tau}\left( \frac{m}{\sqrt{2}i\pi}\int_0^\pi x^i\cos(m\sigma)d\sigma + \sqrt{2}\int_0^\pi \pt^i \cos(m\sigma)d\sigma\right) $$
Et pour $m=0$:
$$a_0^i=\sqrt{2}\int_0^\pi\pt^i d\sigma$$
$$\int_0^\pi x^id\sigma=\pi(q_0^i+\sqrt{2}a_0^i\tau)$$
$$\Rightarrow q_0^i=\int_0^\pi(\frac{x^i}{\pi}- 2\tau \pt^i)d\sigma$$
\textbf{Crochets de Poisson}:\\
\begin{flalign*}
\{a_n^i,a_m^j\}&=e^{in\tau}e^{jm\tau}\left\{\int_0^\pi\left( \frac{nx^i}{\sqrt{2}i\pi}+\sqrt{2}\pt^i\right) \cos(n\sigma)d\sigma,\int_0^\pi\left(\frac{mx^j}{\sqrt{2}i\pi}+\sqrt{2}\pt^j\right)\cos(m\sigma')d\sigma' \right\}
\end{flalign*}
Comme $x^i(\sigma)$ et $\pt^i(\sigma)$ sont canoniquement conjugués, seuls les termes croisés seront non nuls:
\begin{flalign*}
\{a_n^i,a_m^j\}&=\frac{-i}{\pi}e^{in\tau}e^{jm\tau}\int_0^\pi d\sigma\int_0^\pi d\sigma'\cos(n\sigma)\cos(m\sigma')\left(\{nx^i(\sigma),\pt^j(\sigma')\} + \{ \pt^i(\sigma) ,mx^j(\sigma')\}\right)\\
&= \frac{-i}{\pi}e^{in\tau}e^{jm\tau}\int_0^\pi d\sigma\int_0^\pi d\sigma'\cos(n\sigma)\cos(m\sigma')\left(n\delta^{ij}\delta(\sigma-\sigma') - m\delta^{ij}\delta(\sigma-\sigma')\right)\\
&= \frac{-i}{\pi}e^{in\tau}e^{im\tau}\delta^{ij}\int_0^\pi d\sigma\cos(n\sigma)\cos(m\sigma)(n - m)\\
&= \frac{i(m-n)}{\pi}e^{in\tau}e^{im\tau}\delta^{ij}\left[\frac{\pi}{2}(\delta_{n,m}+\delta_{n,-m})\right]\\
&=\left[\frac{i(m-m)}{2}e^{im\tau}e^{im\tau}\delta^{ij}\delta_{n,m}+\frac{i(m-(-m))}{2}e^{-im\tau}e^{im\tau}\delta^{ij}\delta_{n,-m}\right]\\
&=im\delta^{ij}\delta_{n,-m}=-in\delta^{ij}\delta_{n,-m}\\
\{a_0^i,a_0^j\}&=2\int_0^\pi d\sigma\int_0^\pi d\sigma'\{\pt^i(\sigma),\pt^j(\sigma')\}=0\\
\{q_0^i,a_0^j\}&=\left\{\frac{1}{\pi}\int_0^\pi  (x^i(\sigma)-2\tau\pt^i(\sigma))d\sigma,\sqrt{2}\int_0^\pi \pt^j(\sigma')d\sigma'\right\}=\frac{\sqrt{2}}{\pi}\int_0^\pi d\sigma \int_0^\pi d\sigma' \{x^i(\sigma),\pt^j(\sigma')\}\\
&=\frac{\sqrt{2}}{\pi}\delta^{ij}\int_0^\pi d\sigma \int_0^\pi d\sigma'\delta(\sigma-\sigma')=\sqrt{2}\delta^{ij}\\
\{q_0^i,q_0^j\}&=\left\{\frac{1}{\pi}\int_0^\pi  (x^i(\sigma)-2\tau\pt^i(\sigma))d\sigma,\frac{1}{\pi}\int_0^\pi  (x^j(\sigma')-2\tau\pt^j(\sigma'))d\sigma'\right\}\\
&=\frac{-2\tau}{\pi}\int_0^\pi d\sigma\int_0^\pi d\sigma'\left(\{  x^i(\sigma),\pt^j(\sigma')\} +\{\pt^i(\sigma),x^j(\sigma')\}\right)=0\\
\{q_0^-,a_0^+\}&=\{q_{0-},\sqrt{2}\CP_+\}=-\sqrt{2}
\end{flalign*}
Calculons à présent les crochets de Poisson des modes dépendants:
\begin{flalign*}
\{L_n,L_m\}=\frac{1}{4}\sum_{k=-\infty}^{k=+\infty}\sum_{l=-\infty}^{l=+\infty}&\{\vec{a}_{-k} \cdot \vec{a}_{n+k},\vec{a}_{-l} \cdot \vec{a}_{m+l}\}=\frac{1}{4}\sum_{k=-\infty}^{k=+\infty}\sum_{l=-\infty}^{l=+\infty}\sum_{i=1}^{D-2}\sum_{j=1}^{D-2}\{a^i_{-k}a^i_{n+k},a^j_{-l}a^j_{m+l}\}\\
\{a^i_{-k} \cdot a^i_{n+k},a^j_{-l} \cdot a^j_{m+l}\}=& a^i_{-k}\{a^i_{n+k},a^j_{-l}\}a^j_{m+l}+a^i_{-k}\{a^i_{n+k},a^j_{m+l}\}a^j_{-l}\\
+& a^i_{n+k}\{a^i_{-k},a^j_{-l}\}a^j_{m+l}+a^i_{n+k}\{a^i_{-k},a^j_{m+l}\}a^j_{-l}\\
=& a^i_{-k}(-i(n+k)\delta_{n+k,l}\delta^{ij})a^j_{m+l}+a^i_{-k}(-i(n+k)\delta_{n+k,-m-l}\delta^{ij})a^j_{-l}\\
+& a^i_{n+k}(-i(-k)\delta_{-k,l}\delta^{ij})a^j_{m+l}+a^i_{n+k}(-i(-k)\delta_{-k,-m-l}\delta^{ij})a^j_{-l}\\
\end{flalign*}
Donc
\begin{flalign*}
\{L_n,L_m\}=\frac{1}{4}\sum_{k=-\infty}^{k=+\infty}\sum_{l=-\infty}^{l=+\infty}\sum_{i=1}^{D-2}\sum_{j=1}^{D-2}& a^i_{-k}(-i(n+k)\delta_{n+k,l}\delta^{ij})a^j_{m+l}+a^i_{-k}(-i(n+k)\delta_{n+k,-m-l}\delta^{ij})a^j_{-l}\\
&+a^i_{n+k}(-i(-k)\delta_{-k,l}\delta^{ij})a^j_{m+l}+a^i_{n+k}(-i(-k)\delta_{-k,-m-l}\delta^{ij})a^j_{-l}
\end{flalign*}
\begin{flalign*}
&=\frac{1}{4}\sum_{k=-\infty}^{k=+\infty}\sum_{i=1}^{D-2} \left[ -i(n+k)a^i_{-k}a^i_{m+n+k}-i(n+k)a^i_{-k}a^i_{m+n+k}+ika^i_{n+k}a^i_{m-k}+ika^i_{n+k}a^i_{m-k}\right]\\
&=\frac{1}{2}\sum_{k=-\infty}^{k=+\infty}\sum_{i=1}^{D-2} -i(n+k)a^i_{-k}a^i_{m+n+k}+\frac{1}{2}\sum_{k=-\infty}^{k=+\infty}\sum_{i=1}^{D-2}ika^i_{n+k}a^i_{m-k}\\
&=\frac{1}{2}\sum_{k=-\infty}^{k=+\infty}\sum_{i=1}^{D-2} -i(n+k)a^i_{-k}a^i_{m+n+k}+\frac{1}{2}\sum_{k'=-\infty}^{k=+\infty}\sum_{i=1}^{D-2}i(k'+m)a^i_{n+m+k'}a^i_{-k'}\\
&=\frac{1}{2}\sum_{k=-\infty}^{k=+\infty}\sum_{i=1}^{D-2} i(m-n)a^i_{-k}a^i_{m+n+k}=i(m-n)L_{m+n}
\end{flalign*}
\begin{flalign*}
\{L_n,a^j_m\}&=\frac{1}{2}\sum_{k=-\infty}^{+\infty}\sum_{i=1}^{D-2}\{a^i_{-k}a^i_{n+k},a^j_m\}=\frac{1}{2}\sum_{k=-\infty}^{+\infty}\sum_{i=1}^{D-2}\left[ a^i_{-k}\{a^i_{n+k},a^j_m\}+\{a^i_{-k},a^j_m\}a^i_{n+k}\right]\\
&=\frac{1}{2}\sum_{k=-\infty}^{+\infty}\sum_{i=1}^{D-2}\left[ a^i_{-k}(-i(n+k)\delta_{n+k,-m}\delta_{i,j})+(-i(-k)\delta_{-k,-m}\delta_{i,j}a^i_{n+k}\right]\\
&=\frac{1}{2}\left[ima^j_{m+n}+ima^j_{n+m}\right]=ima^j_{m+n}
\end{flalign*}
\begin{flalign*}
\{q_0^i,L_{m}\}&=\frac{1}{2}\sum_{k=-\infty}^{+\infty}\sum_{j=1}^{D-2}\{q_0^i,a_{-k}^j \cdot a_{m+k}^{j}\}=\frac{1}{2}\sum_{k=-\infty}^{+\infty}\sum_{j=1}^{D-2}\left[\{q_0^i,a_{-k}^j\}a_{m+k}^{j}+a_{-k}^j\{q_0^i,a_{m+k}^{j}\}\right]\\
&=\frac{1}{2}\sum_{k=-\infty}^{+\infty}\sum_{j=1}^{D-2}\left[\sqrt{2}\delta_{0,-k}\delta^{ij}a_{n+k}^{j}+a_{-k}^j\sqrt{2}\delta_{0,m+k}\delta^{ij}\right]=\sqrt{2}a_{m}^i
\end{flalign*}

\end{document}