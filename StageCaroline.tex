 % !TeX spellcheck = en_US
\documentclass[a4paper,12pt]{article}
\def\xmu{x^\mu}
\def\vecx{\vec{x}}
\def\CP{\mathcal{P}}
\def\CL{\mathcal{L}}
\def\pt{P_\tau}
\def\vecpt{\vec{\pt}}
\newcommand{\Mup}[1]{M^{#1}}
\newcommand{\sumnotzero}[1]{\sum_{\substack{#1=-\infty \\ #1\neq 0}}^{+\infty}}
\usepackage[utf8]{inputenc}
\usepackage{amsmath}
\usepackage{amssymb}
\usepackage[a4paper]{geometry}
\newcommand{\norm}[1]{\left\lVert#1\right\rVert}
\geometry{hscale=0.80,vscale=0.80,centering}
\title{Rapport de Stage: Etude classique de la corde non massive relativiste}
\author{Caroline Jonas}
\date{Mai 2017}

\begin{document}
\maketitle

\part*{Introduction}
Le but de ce travail est d'étudier l'article "Quantum dynamics of a massless relativistic string"[\ref{article}].\\
Cet article se divise en deux parties. Dans un premier temps, il traite une corde relativiste et sans masse de façon classique en développant les équations du mouvement et le formalisme Hamiltonien.\\
Dans un second temps, les résultats obtenus sont transposés à la mécanique quantique par le principe de correspondance. Dans cette approche quantique, la covariance de la théorie fixe la dimension de l'espace temps à $D=26$.
Nous allons ici étudier et redémontrer les résultats de l'approche classique.
 
\part*{Approche Classique}
\section{Définition de la corde}
Une corde est une courbe dans l'espace dont la forme et la position changent au cours du temps.

Dans l'espace-temps, elle est décrite par une surface à deux dimensions:
$x^{\mu}(\sigma,t)$ ($\sigma$ est un paramètre le long de la corde, et t est le temps).
Pour que le problème soit physique, on impose que chaque point de cette surface se déplace à une vitesse $v\leq c$, c'est-à-dire qu'en chaque point de la surface, il existe un déplacement infinitésimal le long de la surface de type-temps ou de type-lumière. On dit que cette surface est de type-temps.
\section{Paramétrisation générale}
On paramétrise la surface par $x^{\mu}=x^{\mu}(\sigma,\tau)$. On veut calculer l'aire de la surface en termes des coordonnées $\sigma$ et $\mu$.
Soit une surface $\Sigma$ de dimension n dans une variété $M$ de dimension $D>n$ et soit une métrique $g_{\mu\nu}$ dans $M$ sur les coordonnées $\xmu$. Alors il existe une métrique induite $\gamma_{\alpha\beta}$ sur $\Sigma$:\\
\begin{flalign*}
&\mbox{On paramétrise la surface } \Sigma \mbox{ par les coordonnées } \xi^\alpha \mbox{ telles que } \xmu=\xmu(\xi^\alpha).\\
&\mbox{Alors	}g_{\mu\nu}d\xmu dx^\nu=\gamma_{\alpha\beta}d\xi^\alpha d\xi^\beta \Rightarrow \gamma_{\alpha\beta}=g_{\mu\nu}\partial_\alpha\xmu\partial_\beta x^\nu
\mbox{	et l'aire de la surface $\Sigma$ s'écrit:}\\
& A=\int d^n\xi\sqrt{-\gamma}=\sqrt{-det(\partial_\alpha x\partial_\beta x)}
\end{flalign*}
Pour deux coordonnées $(\sigma,\tau)$, $$A=\int d\tau d\sigma \sqrt{-\gamma_{\tau\tau}\gamma_{\sigma\sigma}+\gamma^2_{\tau\sigma}}=\int d\tau d\sigma \sqrt{(\partial_\tau x \partial_\sigma x)^2-(\partial_\tau x)^2(\partial_\sigma x)^2}$$
On impose les contraintes suivantes sur les paramètres:
\begin{enumerate}
\item Les bouts de la corde correspondent à $\sigma=0$ et $\sigma=\pi$
\item Les configurations initiales et finales sont n'importes quelles courbes $x^{\mu}_{i}(\sigma)$ et $x^{\mu}_{f}(\sigma)$  de type espace qui peuvent être reliées par une surface $x^{\mu}(\sigma,\tau)$ de type temps:
autrement dit, $\left( \frac{\partial x^{\mu}}{\partial \tau}\right)^{2}\leq 0 $ et $\left( \frac{\partial x^{\mu}}{\partial \sigma}\right)^{2}\geq 0 $ (car la métrique utilisée est la métrique de Minkowski constante telle que $g^{00}=-1$).
\end{enumerate}
\section{Equations du mouvement}
On trouve les équations du mouvement par le principe de moindre action. L'action choisie est appelée \textit{action de Nambu-Goto}, c'est la plus simple action qui est directement proportionnelle à la surface:
\begin{equation}
S=\frac{-1}{2\pi \alpha'\hbar c^{2}}\int_{\tau_{i}}^{\tau_{f}}\int_{0}^{\pi}d\tau d\sigma \sqrt{(\partial_\tau x \partial_\sigma x)^2-(\partial_\tau x)^2(\partial_\sigma x)^2}
\end{equation} 
La constante à l'avant n'est pas importante et sert juste aux unités. On pose donc $\alpha '=c=\hbar =1$, et ainsi toutes les grandeurs physique du problème sont sans dimension.\\
\textbf{Principe de Moindre Action:} 

\begin{equation}
0=\delta S
\Rightarrow
0=\int_{\tau_{i}}^{\tau_{f}}d\tau\int_{0}^{\pi}d\sigma\delta \CL \label{action}
\end{equation}
où $\CL$ est la densité Lagrangienne et vaut: 
$$\CL=\frac{-1}{2\pi}\sqrt{\left( \frac{\partial x}{\partial \tau}\frac{\partial x}{\partial \sigma}\right)^{2}-\left( \frac{\partial x}{\partial \tau}\right) ^{2}\left( \frac{\partial x}{\partial \sigma}\right) ^{2}}$$ $\Leftrightarrow$ $\CL$ est une fonctionnelle de $x$.\\
Pour simplifier les notations on écrit par la suite: 
$$ \frac{\partial x^{\mu}}{\partial \sigma}\equiv x'^{\mu} \mbox{ et } \frac{\partial x^{\mu}}{\partial \tau}\equiv \dot{x^{\mu}}$$
Donc:
\begin{equation*}
\delta \CL= \frac{\partial \CL}{\partial \dot{\xmu}}\dot{\delta \xmu}+\frac{\partial \CL}{\partial x'^{\mu}}\delta  x'^{\mu}\\
=\frac{\partial \CL}{\partial \dot{x^{\mu}}}\frac{\partial\delta x^{\mu}}{\partial \tau}+\frac{\partial \CL}{\partial x'^{\mu}}\frac{\partial\delta x^{\mu}}{\partial \sigma}
\end{equation*}
En remplaçant dans \eqref{action},
\begin{flalign*}
0=\delta S&=\int_{\tau_{i}}^{\tau_{f}}d\tau\int_{0}^{\pi}d\sigma\left( \frac{\partial \CL}{\partial \dot{x^{\mu}}}\frac{\partial\delta x^{\mu}}{\partial \tau}+\frac{\partial \CL}{\partial x'^{\mu}}\frac{\partial\delta x^{\mu}}{\partial \sigma}\right)\\
&=\int_{0}^{\pi}d\sigma \Big[\frac{\partial \CL}{\partial \dot{x^{\mu}}}\delta x^{\mu}\Big]^{\tau_{f}}_{\tau_{i}}+\int_{\tau_{i}}^{\tau_{f}}d\tau \Big[\frac{\partial \CL}{\partial x'^{\mu}}\delta x^{\mu}\Big]^{\pi}_{0}-\int_{\tau_{i}}^{\tau_{f}}d\tau\int_{0}^{\pi}d\sigma\left( \frac{\partial}{\partial \tau}\frac{\partial \CL}{\partial \dot{x^{\mu}}}+\frac{\partial}{\partial \sigma}\frac{\partial \CL}{\partial x'^{\mu}}\right)\delta x^{\mu}
\end{flalign*}
Ceci est valable $\forall \delta x^{\mu}(\sigma,\tau)$ tel que 
$$\delta x^{\mu}(\sigma,\tau_{i})=\delta x^{\mu}(\sigma,\tau_{f})=0$$
$\Rightarrow$ le premier terme de $\delta S$ est toujours nul.\\
Et donc finalement,
\begin{equation}
\delta S=0 \Leftrightarrow
\left\lbrace
	\begin{aligned}
    \frac{\partial}{\partial \tau}\frac{\partial \CL}{\partial \dot x^{\mu}}+\frac{\partial}{\partial \sigma}\frac{\partial \CL}{\partial  x'^{\mu}}=0\\
    \frac{\partial \CL}{\partial x'^{\mu}}(0, \tau)=\frac{\partial \CL}{\partial x'^{\mu}}(\pi, \tau)=0
    \end{aligned}\label{motion1}
    \right.  
\end{equation}
Le système \eqref{motion1} est le système d'équations classiques du mouvement de la corde.\\
\paragraph{Calcul explicite de $\frac{\partial \CL}{\partial x'_{\mu}}$ et $\frac{\partial \CL}{\partial \dot x_{\mu}}$:}
Dans la suite, nous allons utiliser les équations du mouvement sous leur forme covariante:
$$\frac{\partial}{\partial \tau}\frac{\partial \CL}{\partial \dot x_{\mu}}+\frac{\partial}{\partial \sigma}\frac{\partial \CL}{\partial  x'_{\mu}}=0$$
Comme ces expressions vont nous être très souvent utiles, nous allons calculer explicitement $\frac{\partial \CL}{\partial x'_{\mu}}$ et $\frac{\partial \CL}{\partial \dot x_{\mu}}$ en terme de $\dot{x}$ et $x'$.\\
On a pour l'instant  $$\CL=\frac{-1}{2\pi}\sqrt{(x'\dot{x})^2-x'^2\dot{x}^2}=\frac{-1}{2\pi}\sqrt{(\dot{\xmu}x'_\mu)^2-(\dot{\xmu}\dot{x_\mu})(x'^\nu x'_\nu)}$$
Ainsi,
\begin{equation}
	\left\lbrace 
	\begin{aligned}
	\frac{\partial \CL}{\partial x'_\mu }=\frac{-1}{2\pi.2}\frac{2\dot{\xmu}(x'\dot{x})-2x'^\mu (\dot{x}^2)}{\sqrt{(x'\dot{x})^2-x'^2\dot{x}^2}}\\
	\frac{\partial \CL}{\partial \dot{x_\mu}}=\frac{-1}{2\pi.2}\frac{2x'^\mu(x'\dot{x})-2\dot{\xmu} (x'^2)}{\sqrt{(x'\dot{x})^2-x'^2\dot{x}^2}}\\
	\end{aligned} 
	\right. 
	\Leftrightarrow\\
	\left\lbrace
	\begin{aligned}
	\frac{\partial \CL}{\partial x'_\mu }=\frac{1}{2\pi}\frac{x'^\mu (\dot{x}^2)-\dot{\xmu}(x'\dot{x})}{\sqrt{(x'\dot{x})^2-x'^2\dot{x}^2}}\\
	\frac{\partial \CL}{\partial \dot{x_\mu}}=\frac{1}{2\pi}\frac{\dot{\xmu} (x'^2)-x'^\mu(x'\dot{x})}{\sqrt{(x'\dot{x})^2-x'^2\dot{x}^2}}\\
	\end{aligned}
	\right.\label{mvt}
\end{equation}
\section{Quantités conservées}
Comme le Lagrangien dépend uniquement du produit scalaire et des normes minkowskiennes de $\dot{x^{\mu}}$ et $x'^{\mu}$, il est invariant sous le groupe des transformations de Poincaré.

\textsc{Rappel}: Le groupe de Poincaré est l'ensemble des transformations de l'espace temps qui préservent la distance minkowskienne:
$$ x^{\mu}\rightarrow\Lambda_{\nu}^{\mu}x^{\nu}+a^{\mu}$$ 
Où $\Lambda_{\nu}^{\mu}$ est une transformation de Lorentz et $a^{\mu}$ est une translation dans l'espace temps.\\
\newpage
\paragraph{Théorème de Noether en théorie classique des champs:[\ref{QFT}]}
Le théorème de Noether établit un lien entre les symétries d'un système physique et les lois de conservation s'appliquant dans ce système.\\
Soit un ensemble de champs différentiables $\phi$ définis sur tout l'espace-temps. L'action est l'intégrale sur l'espace et le temps de la densité lagrangienne $\CL$:
$$\mathcal{S}=\int\CL(\phi,\partial_\mu \phi,\xmu)d^4x$$
Une transformation infinitésimale continue du champ $\phi$ peut s'écrire:
$\phi \rightarrow \phi' = \phi + \alpha \Delta \phi$ où $\alpha$ est un paramètre infinitésimal et $\Delta\phi$ est une déformation du champ $\phi$.
Cette transformation génère une symétrie physique si elle laisse les équations du mouvement invariantes, autrement dit si elle laisse $\CL$ invariant à une divergence près: $$\CL \rightarrow \CL + \alpha \partial_\mu J^\mu$$
Soit une variation infinitésimale de $\CL$:
\begin{flalign*}
\alpha\Delta\CL=&\alpha\frac{\partial\CL}{\partial \phi}\Delta\phi + \frac{\partial \CL}{\partial(\partial_{\mu}\phi)}\partial_{\mu}(\alpha\Delta\phi)\\
=&\alpha\frac{\partial\CL}{\partial \phi}\Delta\phi + \alpha\partial_{\mu}\left(\frac{\partial \CL}{\partial(\partial_{\mu}\phi)}\Delta\phi\right)-\alpha\partial_{\mu}\left(\frac{\partial \CL}{\partial(\partial_{\mu}\phi)}\right)\Delta\phi
\end{flalign*}
Par les équations d'Euler-Lagrange, 
$$\frac{\partial\CL}{\partial \phi} -\partial_{\mu}\left(\frac{\partial \CL}{\partial(\partial_{\mu}\phi)}\right)=0$$
Donc
$$\alpha\Delta\CL=\alpha\partial_{\mu}\left(\frac{\partial \CL}{\partial(\partial_{\mu}\phi)}\Delta\phi\right)$$
Si la transformation est une symétrie, alors $\alpha\Delta\CL=\alpha\partial_{\mu}J^\mu$\\
Donc 
$$\partial_{\mu}\left(\frac{\partial \CL}{\partial(\partial_{\mu}\phi)}\Delta\phi - J^\mu\right)=0$$
$\Rightarrow j^\mu=\frac{\partial \CL}{\partial(\partial_{\mu}\phi)}\Delta\phi - J^\mu$ est un courant conservé$\Leftrightarrow \partial_\mu j^\mu=0$. Cette conservation est aussi équivalente au fait que la charge $$\mathcal{Q}\equiv\int_{space}j^0 d^3 x$$ est constante dans le temps.
$$------------------------------------------------------$$
Dans notre cas, le Lagrangien est laissé invariant par les transformations de Lorentz $$ x^{\mu}\rightarrow\Lambda_{\nu}^{\mu}x^{\nu}$$ et par les translations dans l'espace-temps $$ x^{\mu}\rightarrow \xmu+a^{\mu}$$
Par le théorème de Noether, chacune de ces transformations va donner lieu à un courant conservé et une charge constante.

\subsection{Translations:}

Le courant associé aux translations est défini sur la surface par:  
$$P^{\mu}_{\tau}=\frac{\partial \CL}{\partial \dot{x_{\mu}}},			 P^{\mu}_{\sigma}=\frac{\partial \CL}{\partial x'_{\mu}}$$
On appelle les $P^\mu_i$ les courants d'impulsion.
Si on remplace ces quantités dans les équations du mouvement, on trouve:
$$\frac{\partial P^{\mu}_{\tau}}{\partial \tau}+\frac{\partial P^{\mu}_{\sigma}}{\partial \sigma}=0$$
 $$P^{\mu}_{\sigma}(0,\tau)=P^{\mu}_{\sigma}(\pi,\tau)=0$$
Ce qui revient en fait à la conservation du courant d'impulsion sur la surface.
La charge constante dans le temps est appelée \textbf{impulsion totale de la corde} et s'écrit:
$$\mathcal{P^{\mu}}=\int_{0}^{\pi}P^{\mu}_{\tau}d\sigma$$
\subsection{Transformations de Lorentz}
Le courant associé aux transformations de Lorentz est:
$M_{i}^{\mu\nu}\equiv x^{\mu}P^{\nu}_{i}-x^{\nu}P^{\mu}_{i}$\\
On l'appelle le courant de moment angulaire.
Comme pour l'impulsion, les équations du mouvement expriment la conservation locale du moment angulaire:
$$\frac{\partial M_{\sigma}^{\mu\nu}}{\partial \sigma}+\frac{\partial M_{\tau}^{\mu\nu}}{\partial \tau}=x^{\mu}\left( \frac{\partial P^{\nu}_{\tau}}{\partial \tau}+\frac{\partial P^{\nu}_{\sigma}}{\partial \sigma}\right) +\left( \dot{x^{\mu}}P^{\nu}_{\tau}+x'^{\mu}P^{\nu}_{\sigma}\right) -x^{\nu}\left( \frac{\partial P^{\mu}_{\tau}}{\partial \tau}+\frac{\partial P^{\mu}_{\sigma}}{\partial \sigma}\right) -\left( \dot{x^{\nu}}P^{\mu}_{\tau}+x'^{\nu}P^{\mu}_{\sigma}\right)$$
Par les équations du mouvements exprimées en fonction de $P$, nous voyons directement que le premier et le troisième terme sont nuls. 
Ré-écrivons les termes 2 et 4 en fonctions de $\dot{x}$ et $x'$ grâce à l'équation \eqref{mvt}:
\begin{equation}
	\left\lbrace
	\begin{aligned}
	\dot{x^{\mu}}P^{\nu}_{\tau}=\frac{1}{2\pi}\frac{\dot{\xmu}\dot{x^\nu}(x')^2-\dot{\xmu}x'^\nu(x'\dot{x})}{\sqrt{(x'\dot{x})^2-x'^2\dot{x}^2}}\\
	x'^{\mu}P^{\nu}_{\sigma}=\frac{1}{2\pi}\frac{x'^\mu x'^\nu (\dot{x})^2-x'^{\mu}\dot{x^\nu}(x'\dot{x})}{\sqrt{(x'\dot{x})^2-x'^2\dot{x}^2}}\\
	-\dot{x^{\nu}}P^{\mu}_{\tau}=-\frac{1}{2\pi}\frac{\dot{x^\nu}\dot{\xmu}(x')^2-\dot{x^\nu}x'^\mu(x'\dot{x})}{\sqrt{(x'\dot{x})^2-x'^2\dot{x}^2}}\\
	-x'^{\nu}P^{\mu}_{\sigma}=-\frac{1}{2\pi}\frac{x'^{\nu}x'^\mu (\dot{x})^2-x'^{\nu}\dot{\xmu}(x'\dot{x})}{\sqrt{(x'\dot{x})^2-x'^2\dot{x}^2}}
	\end{aligned}
	\right.
\end{equation}
On voit que la somme de ces quatre termes est nulle. 
Les équations du mouvement en terme des moments angulaires sont donc:
$$\frac{\partial M_{\sigma}^{\mu\nu}}{\partial \sigma}+\frac{\partial M_{\tau}^{\mu\nu}}{\partial \tau}=0$$
Et les conditions de bords sont:
$$M_{\sigma}^{\mu\nu}(0,\tau)=M_{\sigma}^{\mu\nu}(\pi,\tau)=0$$
La charge constante est ici appelée \textbf{moment angulaire total de la corde} et s'écrit: 
$$\mathcal{M^{\mu\nu}}=\int_{0}^{\pi}M^{\mu\nu}_{\tau}d\sigma$$
\section{Spécification de la paramétrisation}
L'action de Nambu-Goto laisse une liberté dans le choix de la paramétrisation, car les équations de mouvement gardent la même forme pour les paramètres $(\tilde{\sigma},\tilde{\tau})$ à condition que les courbes $x(\tilde{\sigma}=0,\tilde{\tau}), x(\tilde{\sigma}=\pi,\tilde{\tau})$ coïncident avec les courbes $x(\sigma=0,\tau), x(\sigma=\pi,\tau)$.
Par conséquent, il faut choisir cette paramétrisation avant de résoudre les équations.\\
Premièrement, on montre que les contraintes suivantes sur la paramétrisation:
\begin{flalign}
& x'. \dot{x}=0\label{contrainte1}\\
&\left( x'\right) ^{2}+\left(\dot{x}\right) ^{2}=0\label{contrainte2}
\end{flalign}
simplifient les équations du mouvement.\\ Secondement, on justifie ces contraintes (et d'autres) par des arguments physiques. 
\subsection{Simplification des équations}
En utilisant la contrainte \eqref{contrainte1} sur la paramétrisation, on peut ré-écrire l'équation \eqref{mvt} comme:
\begin{equation*}
	\left\lbrace
	\begin{aligned}
	\frac{\partial \CL}{\partial x'_\mu }=\frac{1}{2\pi}\frac{x'^\mu (\dot{x})^2}{\sqrt{-x'^2\dot{x}^2}}\\
	\frac{\partial \CL}{\partial \dot{x_\mu}}=\frac{1}{2\pi}\frac{\dot{\xmu} (x')^2}{\sqrt{-x'^2\dot{x}^2}}\\
	\end{aligned}
	\right.
\end{equation*}
Par la contrainte \eqref{contrainte2}, on a que:
$$\frac{\dot{x}^2}{\sqrt{-x'^2\dot{x}^2}}=\frac{\dot{x}^2}{\sqrt{\dot{x}^4}}=-1$$ car on sait que $(\dot{x}^2)\leq0$.
Et similairement,
$$\frac{x'^2}{\sqrt{-x'^2\dot{x}^2}}=\frac{x'^2}{\sqrt{x'^4}}=+1$$
car on sait que $(x'^2)\geq0$.
Donc
\begin{equation*}
	\left\lbrace
	\begin{aligned}
	\frac{\partial \CL}{\partial x'_\mu }=-\frac{1}{2\pi}x'^\mu\\
	\frac{\partial \CL}{\partial \dot{x}^\mu}=+\frac{1}{2\pi}\dot{x}^\mu\\
	\end{aligned}
	\right.
\end{equation*}
Et les équations du mouvement deviennent:
$$\frac{\partial}{\partial \tau}\frac{\partial \CL}{\partial \dot{x}_{\mu}}+\frac{\partial}{\partial \sigma}\frac{\partial \CL}{\partial  x'_{\mu}}=0 \Leftrightarrow \frac{1}{2\pi}\frac{\partial}{\partial \tau}\left( \dot{x}^\mu\right)+\frac{-1}{2\pi}\frac{\partial}{\partial \sigma}\left( x'^\mu\right) $$
$$\Leftrightarrow \frac{1}{2\pi}\left[ \frac{\partial^2 x^\mu}{\partial \tau^2}-\frac{\partial^2 x^\mu}{\partial \sigma^2}\right] =0 \Leftrightarrow \left( \frac{\partial^2}{\partial \tau^2}-\frac{\partial^2 }{\partial \sigma^2}\right) \xmu(\sigma,\tau)=0$$
c'est-à-dire l'équation d'onde classique.
\subsection{Arguments physiques pour le choix de la paramétrisation}
On veut que $\tau$ soit identifié à une coordonnée temporelle et que $\sigma$ soit une quantité qui varie toujours de $0$ à $\pi$ indépendamment du temps.
On pose donc les deux égalités suivantes:
$$n.x=2(n.\mathcal{P})\tau$$
$$(n.\mathcal{P})\sigma=\pi\int_0^\sigma d\sigma (n.P_\tau)$$
où $n$ est un vecteur à D dimensions constant tel que $n^2\leq0$.
Ainsi, comme $\CP$ est une constante (par conservation du courant de Noether), 
$n.P_\tau$ est également une constante et:
$$n.P_\tau=\frac{n.\mathcal{P}}{\pi}$$\\
Les équations du mouvement nous donnent:
\begin{equation*}
\frac{\partial P^{\mu}_{\tau}}{\partial \tau}+\frac{\partial P^{\mu}_{\sigma}}{\partial \sigma}=0 \Leftrightarrow \frac{\partial n.P_{\tau}}{\partial \tau}+\frac{\partial n.P_{\sigma}}{\partial \sigma}=0 \Leftrightarrow \frac{\partial n.P_{\sigma}}{\partial \sigma}=0 \mbox{ car } n.P_\tau \mbox{ est constant.}
\end{equation*}

\begin{equation*}
\rightarrow\mbox{Comme les conditions au bord donnent: }  P_\sigma^\mu(0,\tau)=P_\sigma^\mu(\pi,\tau)=0\mbox{, on a que } n.P_\sigma=0 \mbox{ }\forall (\sigma,\tau)
\end{equation*}
D'autre part, 
$$n.P_\sigma=n_\mu \frac{\partial \CL}{\partial x'_\mu }, n.P_\tau=n_\mu.\frac{\partial \CL}{\partial \dot{x_\mu}}$$
En utilisant l'équation \eqref{mvt}:
\begin{equation*}
	\left\lbrace
	\begin{aligned}
	\frac{\partial \CL}{\partial x'_\mu }=\frac{1}{2\pi}\frac{x'^\mu (\dot{x}^2)-\dot{\xmu}(x'\dot{x})}{\sqrt{(x'\dot{x})^2-x'^2\dot{x}^2}}\\
	\frac{\partial \CL}{\partial \dot{x_\mu}}=\frac{1}{2\pi}\frac{\dot{\xmu} (x'^2)-x'^\mu(x'\dot{x})}{\sqrt{(x'\dot{x})^2-x'^2\dot{x}^2}}\\
	\end{aligned}
	\right.
	\Rightarrow\\
	\left\lbrace
	\begin{aligned}
	n.P_\sigma=\frac{1}{2\pi}\frac{n.x' (\dot{x}^2)-n.\dot{x}(x'\dot{x})}{\sqrt{(x'\dot{x})^2-x'^2\dot{x}^2}}\\
	n.P_\tau=\frac{1}{2\pi}\frac{n.\dot{x} (x'^2)-n.x'(x'\dot{x})}{\sqrt{(x'\dot{x})^2-x'^2\dot{x}^2}}
	\end{aligned}
	\right.
\end{equation*}
Or, $n.x=2(n.\mathcal{P})\tau$ où $n$ et $\mathcal{P}$ sont des constantes,
donc $n.\dot{x}=2(n.\mathcal{P})$ et $n.x'=0$,
Par conséquent, 
\begin{equation*}
	\left\lbrace
	\begin{aligned}
	0=n.P_\sigma=\frac{1}{2\pi}\frac{-2(n.\mathcal{P})(x'\dot{x})}{\sqrt{(x'\dot{x})^2-x'^2\dot{x}^2}}\\
	\frac{n.\mathcal{P}}{\pi}=n.P_\tau=\frac{1}{2\pi}\frac{2(n.\mathcal{P}) (x'^2)}{\sqrt{(x'\dot{x})^2-x'^2\dot{x}^2}}
	\end{aligned}
	\right.
	\Rightarrow\\
	\left\lbrace
	\begin{aligned}
	(x'\dot{x})=0\\
	\frac{n.\mathcal{P}}{\pi}=\frac{1}{\pi}\frac{(n.\mathcal{P}) (x'^2)}{\sqrt{-x'^2\dot{x}^2}}
	\end{aligned}
	\right.	
\end{equation*}
\begin{equation*}
	\Rightarrow\left\lbrace
	\begin{aligned}
	(x'\dot{x})=0\\
	1=\frac{ (x'^2)}{\sqrt{-x'^2\dot{x}^2}}
	\end{aligned}
	\right.
	\Leftrightarrow\\
	\left\lbrace
	\begin{aligned}
	(x'\dot{x})=0\\
	(-x'^2\dot{x}^2)=(x'^4)
	\end{aligned}
	\right.	
	\Leftrightarrow\\
	\left\lbrace
	\begin{aligned}
	(x'\dot{x})=0\\
	(x'^2)+(\dot{x}^2)=0
	\end{aligned}
	\right.	
\end{equation*}
\textbf{Conclusion:}
\begin{equation}
	\left\lbrace
	\begin{aligned}
		\frac{\partial x}{\partial \sigma}.\frac{\partial x}{\partial \tau}=0, \left( \frac{\partial x}{\partial \sigma}\right) ^2+\left( \frac{\partial x}{\partial \tau}\right) ^2=0\\ 
    	\left( \frac{\partial^2}{\partial  \tau^2}-\frac{\partial^2}{\partial  \sigma^2}\right) \xmu=0\\ 
    	\frac{\partial \xmu}{\partial \sigma}=0 \mbox{  si $\sigma=0,\pi$}\\ 
    	n.x=2(n.\mathcal{P})\tau 
	\end{aligned}
    \right.
\end{equation}
\section{Formalisme Hamiltonien}
Pour passer ensuite à la formulation quantique par le principe de correspondance, on doit construire le formalisme Hamiltonien de ce système. Cependant, les équations du mouvement sont contraintes par \eqref{contrainte1} et \eqref{contrainte2}.
De plus, $\xmu$ et $P_\tau^\mu$ ne peuvent pas être prises comme les variables canoniquement conjuguées car:
\begin{align*}
& n.x=2(n.\mathcal{P})\tau & n.P_\tau=\frac{n.\mathcal{P}}{\pi}&
\end{align*}
Autrement dit, $\xmu$ et $\pt^\mu$ ne sont pas indépendantes l'une de l'autre. Sous contraintes, on a deux méthodes pour obtenir le formalisme Hamiltonien:
\begin{enumerate}
\item On calcule les crochets de Poisson de toutes les variables et puis on impose les contraintes.
\item On résout d'abord explicitement les contraintes en éliminant certaines variables et on calcule les crochets de Poisson seulement sur les variables restantes.
\end{enumerate} 
On va utiliser la deuxième méthode.
\subsection{Variables non contraintes}
On introduit de nouvelles notations et on spécifie $n$ de façon explicite.
\begin{enumerate}
\item $\forall$ D-vecteur $r=(r^0,\vec{r},r^{D-1})$, on définit
$r_\pm=\frac{1}{\sqrt2}(r^0\pm r^{D-1})$
\item $n=\frac{1}{\sqrt2}(1,\vec{0},-1)$, c'est-à-dire que 
$n_+=0$ et $n_-=1$ ( et $n$ est de type-lumière: le temps sur la corde est défini par une direction de type-lumière)
\end{enumerate}
On va résoudre explicitement les contraintes pour fixer certaines des composantes de $\xmu$ et de $P_\tau^\mu$: 
\subsubsection{$n.x=2(n.\mathcal{P})\tau$}\label{point1}
\begin{align*}
& n_0x^0 + \vec{n}.\vec{x} + n_{D-1}x^{D-1}=2\left( n_0\mathcal{P}^0 + \vec{n}.\vec{\mathcal{P}} + n_{D-1}\mathcal{P}^{D-1}\right) \tau\\
\Leftrightarrow &\frac{-1}{\sqrt{2}}x^0 + \frac{-1}{\sqrt{2}}x^{D-1}=2\left( \frac{-1}{\sqrt{2}}\mathcal{P}^0  + \frac{-1}{\sqrt{2}}\mathcal{P}^{D-1}\right) \tau &\Rightarrow\boxed{x_+ = 2 \mathcal{P_+}\tau}
\end{align*}
\subsubsection{$n.P_\tau=\frac{n.\mathcal{P}}{\pi}$}
\begin{align*}
&\Leftrightarrow n_0P_\tau^0 + \vec{n}.\vec{x} + n_{D-1}P_\tau^{D-1}=\frac{1}{\pi}\left( n_0\mathcal{P}^0 + \vec{n}.\vec{\mathcal{P}} + n_{D-1}\mathcal{P}^{D-1}\right)\\
&\Leftrightarrow \frac{-1}{\sqrt{2}}P_\tau^0 + \frac{-1}{\sqrt{2}}P_\tau^{D-1}=\left( \frac{-1}{\sqrt{2}\pi}\mathcal{P}^0  + \frac{-1}{\sqrt{2}}\mathcal{P}^{D-1}\right)
&\Rightarrow\boxed{P_{\tau+} = \frac{\mathcal{P_+}}{\pi}}
\end{align*}
\subsubsection{$(\dot{x}x')=0$}
\begin{equation*}
 x'.P_\tau=0
\Leftrightarrow -x'_0P_\tau^0 +\vec{x'}.\vec{P_\tau} + x'_{D-1}.P_\tau^{D-1} =0
\Leftrightarrow x'^0P_\tau^0 - x'^{D-1}.P_\tau^{D-1}=\vec{x'}.\vec{P_\tau}
\end{equation*}
Or,
\begin{equation*}
x'_+P_{\tau -} + x'_-P_{\tau +}=\frac{1}{2}\left( x'^0 + x'^{D-1}\right) \left(P_\tau^0 - P_\tau^{D-1}\right) + \frac{1}{2}\left( x'^0 - x'^{D-1}\right) \left(P_\tau^0 + P_\tau^{D-1}\right)=x'^0P_\tau^0 - x'^{D-1}.P_\tau^{D-1}
\end{equation*}
De plus, par \eqref{point1} on a que $x_+ = 2 \mathcal{P_+}\tau$ donc $x'_+=\frac{\partial x_+}{\partial \sigma}=0$ car $\mathcal{P}$ est une constante.
$$\Rightarrow\boxed{x'_-P_{\tau +}=\vec{x'}.\vec{P_\tau}}$$
\subsubsection{$(\dot{x}^2)+(x'^2)=0$}
\begin{align*}
&\frac{1}{(2\pi)^2}\left[ -(x'^0)^2+(\vec{x'})^2 + (x'^{D-1})^2\right] +\left[ -(\pt^0)^2 +(\vec{P_\tau})^2 + (\pt^{D-1})^2\right]=0\\
\Leftrightarrow&\frac{1}{(2\pi)^2}(\vec{x'})^2 + (\vec{P_\tau})^2 - \frac{1}{(2\pi)^2}\left( 2x'_+x'_-\right) - (2P_{\tau +}P_{\tau -})=0
\Rightarrow\boxed{\frac{1}{(2\pi)^2}(\vec{x'})^2 + (\vecpt)^2=2P_{\tau +}P_{\tau -}}
\end{align*}
On a trouvé quatre contraintes directement sur les variables $x$ et $P_\tau$. Les deux premières équations fixent $P_{\tau +}$ et $x_+$ en fonction de $\CP_+$.\\
On introduit ensuite \textit{la coordonnée barycentrique}:
$$q_-(\tau)=\frac{1}{\pi}\int_0^\pi d\sigma x_-(\sigma,\tau)$$
Alors, les deux dernières équations nous permettent d'écrire:
\begin{flalign}
& P_{\tau -}(\sigma,\tau)=\frac{\pi}{2\mathcal{P_+}}\left[\frac{1}{(2\pi)^2}(\vec{x'})^2 + (\vec{\pt})^2\right]\label{pt-}\\
& x_-(\sigma,\tau) = q_-(\tau) + \frac{\pi}{\mathcal{P_+}}\int_0^\pi d\sigma'\left( \frac{\sigma'}{\pi}-\theta(\sigma'-\sigma)\right) \vecx'\vecpt\label{x-}
\end{flalign}
\textbf{Démonstration:}
L'équation \eqref{pt-} s'obtient directement en utilisant les 4 contraintes précédentes. Nous montrons que l'équation \eqref{x-} est bien cohérente avec la définition de $q_-(\tau)$:
$$q_-(\tau)=\frac{1}{\pi}\int_0^\pi d\sigma x_-(\sigma,\tau)
\Rightarrow q_-(\tau)=\frac{1}{\pi}\int_0^\pi d\sigma\left[ q_-(\tau) + \frac{\pi}{\mathcal{P_+}}\int_0^\pi d\sigma'\left( \frac{\sigma'}{\pi}-\theta(\sigma'-\sigma)\right) \vecx'\vecpt\right] $$
$q_-(\tau)$ ne dépend pas de $\sigma$ donc l'intégrale du premier terme nous donne $q_-(\tau)$. Il reste à montrer que l'intégrale du second terme donne 0. On utilise la relation (2.7.3): $\frac{\pi}{\CP}\vecx'\vecpt=x'_-$.
On réécrit: 
$$\frac{1}{\pi}\int_0^\pi d\sigma \int_0^\pi d\sigma'\left( \frac{\sigma'}{\pi}-\theta(\sigma'-\sigma)\right) \frac{\pi}{\mathcal{P_+}}\vecx'\vecpt =\frac{1}{\pi}\int_0^\pi d\sigma\left[ \int_0^\pi d\sigma'\left( \frac{\sigma'}{\pi}-\theta(\sigma'-\sigma)\right) \frac{\partial x_-}{\partial \sigma'}\right]$$
$$=\frac{1}{\pi^2}\int_0^\pi d\sigma\Big[\left[\sigma'.x_-(\sigma',\tau)\right]^\pi_0 - \int_0^\pi d\sigma' x_-(\sigma',\tau)\Big] - \frac{1}{\pi}\int_0^\pi d\sigma\int_\sigma^\pi d\sigma'\frac{\partial x_-}{\partial \sigma'}(\sigma',\tau)$$
$$=\frac{1}{\pi^2}\int_0^\pi d\sigma \pi.x_-(\pi,\tau)-\frac{1}{\pi^2}\int_0^\pi d\sigma \pi*q_-(\tau) - \frac{1}{\pi}\int_0^\pi d\sigma\left( x_-(\pi,\tau)-x_-(\sigma,\tau)\right) $$
$$= x_-(\pi,\tau)-q_-(\tau) -  x_-(\pi,\tau)+q_-(\tau) = 0$$
Cette expression pour $x_-(\sigma,\tau)$ est bien cohérente.\\
\textbf{Conclusion:} Nous pouvons  exprimer toutes les composantes de $x$ et $\pt$ en fonction de:
$$\boxed{\vecx, \vecpt, \CP_+, q_-}$$
Ce sont les variables non contraintes sur lesquelles on va calculer les crochets de Poisson. 
\subsection{Crochets de Poisson}
\textsc{Rappel:} Les crochets de Poisson et les équations du mouvement sont reliés par 
$$\dot{f}=\{f,H\}+\frac{\partial f}{\partial \tau}$$
Comme on connait les équations du mouvement, on va en déduire $\dot{\vecx}, \dot{\vecpt}, \dot{\CP_+}, \dot{q_-}$  et en calculant ensuite l'Hamiltonien, on pourra trouver les crochets de Poisson.\\
\textbf{Remarque:} la dérivée partielle par rapport à $\tau$ dans l'expression ci-dessus doit se comprendre comme une dérivée explicite par rapport à $\tau$. La dérivée pointée est ce que nous notions comme la dérivée par rapport à $\tau$ précédemment, c'est-à-dire la dérivée temporelle sur la surface.\\
\paragraph*{Etape 1: calcul des dérivées pointées}
\paragraph{$\vecpt:$}
$\pt^\mu=\frac{\dot{\xmu}}{2\pi}$ et $\left( \ddot{\xmu}-x^{\mu''} \right)=0$
$\Leftrightarrow 2\pi\dot{\pt^\mu}-x''^\mu=0$
$$\Rightarrow \dot{\vecpt}=\frac{1}{2\pi}\vecx''$$
\paragraph{$\vecx$:}
$\pt^\mu=\frac{\dot{\xmu}}{2\pi}$  $$\Rightarrow \dot{\vecx}=2\pi\vecpt$$
\paragraph{$\CP_+$:}
$$\CP_+= cst$$ 
\paragraph{$q_-$:}
$$\dot{q}_-=\frac{1}{\pi}\int_0^\pi d\sigma \dot{x}_-= 2\int_0^\pi d\sigma P_{\tau -}\equiv 2 \CP_-$$
\paragraph*{Etape 2: calcul de l'Hamiltonien.}
On doit exprimer l'Hamiltonien du système. On est tenté pour ce faire de retourner à l'expression du Lagrangien du système et de trouver l'Hamiltonien par transformée de Legendre. Cependant cette méthode n'est plus valable maintenant qu'on a fixé une partie des variables par les contraintes (si on le fait, on obtient des contradictions comme $H=0$).\\
Pour déterminer H, on utilise le fait que H décrit l'évolution temporelle du système:
$$H \leftrightarrow \frac{\partial}{\partial \tau}$$
D'autre part, on sait que $\CP_-$ génère les translations infinitésimales dans la direction de $x_+$: $$\frac{\partial}{\partial x_+}\leftrightarrow \CP_-$$
et que (relation (6.1.1)): $$x_+=2\CP_+.\tau$$ Donc $$H \leftrightarrow \frac{\partial}{\partial \tau}=2\CP_+\frac{\partial}{\partial x_+}=2\CP_+\CP_- $$
Par l'équation (22) et la définition de $\CP_-$, on obtient: $$H=\pi\int_0^\pi\left( \vecpt^2 +\frac{\vecx^{'2}}{(2\pi)^2}\right)d\sigma=\dot{q}_-\CP_+ $$
\paragraph*{Etape 3: calcul des crochets de Poisson}
Les équations du mouvement du système sont:
\begin{equation}
	\left\lbrace
    \begin{aligned}
   		\dot{x}^i=2\pi\pt^i\\
		\dot{\pt}^i=\frac{1}{2\pi}x^{'' i}\\
		\dot{q}_-=2\CP_-\\
		\dot{\CP_+}=0
     \end{aligned}
     \right.
\end{equation}
Les équations de Hamilton sont:
\begin{equation}
	\left\lbrace
    \begin{aligned}
		\dot{q}^i=\{q^i,H\}=\frac{\partial H}{\partial p^i}\\
		\dot{p}^i=\{p^i,H\}=-\frac{\partial H}{\partial q^i}\\
     \end{aligned}
     \right.
\end{equation}
 Montrons que ces équations sont équivalentes pour $q^i=x^i(\sigma)$, $p^i=\pt^i(\sigma)$:\\
\textbf{NB:} L'Hamiltonien est une fonction de $q^i$ et de $p^i$ qui sont considérées indépendantes dans cette formulation. \\
\begin{equation*}
\delta H = \pi\int_0^\pi\left(2\pt^i \delta P_{\tau i} +\frac{2x^{'i}}{(2\pi)^2}\delta x^{'}_{i}\right) d\sigma=\int_0^\pi \left( 2\pi\pt^i(\sigma) \delta P_{\tau i}(\sigma) -\frac{x^{''i}(\sigma)}{2\pi}\delta x_{i}(\sigma)\right)  d\sigma 
\end{equation*}
où la deuxième égalité est une intégration par parties où on a supprimé le terme aux bords (les variations $\delta x$ sont toujours nuls aux bords).\\
Or par définition,
\begin{equation*}
\delta H=\int_0^\pi\left( \frac{\delta H}{\delta \pt^i}\delta \pt^i + \frac{\delta H}{\delta x^i}\delta x^i\right)d \sigma
\end{equation*}
\begin{equation*}
\Rightarrow
\left\lbrace
\begin{aligned}
\frac{\delta H}{\delta \pt^i(\sigma)}=2\pi \pt^i (\sigma)= \dot{x}^i(\sigma)\\
\frac{\delta H}{\delta x^i(\sigma)}= -\frac{x''^i(\sigma)}{2\pi}=-\dot{\pt}^i(\sigma)
\end{aligned}
\right.
\end{equation*}
Par conséquent, $x^i(\sigma)$ et $\pt^i(\sigma)$ sont canoniquement conjuguées:
\begin{equation}
	\begin{aligned}
	\{x^i(\sigma),\pt^j(\sigma')\}=\delta^{ij}\delta(\sigma-\sigma')\\
	\{x^i,x^j\}=\{\pt^i,\pt^j\}=0
	\end{aligned}\label{crp1}
\end{equation}
D'autre part, $$\dot{q}_-=2\CP_-=\frac{H}{\CP_+}$$
Donc $$q_-=q_{0-}+\frac{H}{\CP_+}\tau$$
$\Rightarrow$ $\CP_+$ et $q_{0-}$ sont deux constantes du mouvement qui complètent les variables canoniques. On suppose que 
\begin{equation}
\{q_{0-},\CP_+\}=-1\label{crp2}
\end{equation}
($\equiv$ $\CP_+$ est le générateur des translations infinitésimales dans la direction de $x_-$).
Comme ce sont des constantes du mouvement, on a également:
\begin{equation}
\{\CP_+,\vec{\pt}\}=\{\CP_+,\vec{x}\}=\{q_{0-},\vec{\pt}\}=\{q_{0-},\vec{x}\}=0\label{crp3}
\end{equation}
Les équations \eqref{crp1}, \eqref{crp2} et \eqref{crp3} fixent tous les crochets de Poisson parmi les variables non contraintes.
\subsection{Modes normaux}
Comme il est plus facile de travailler avec des indices discrets, on va utiliser les modes normaux au lieu des variables $x$ et $\pt$.\\
L'équation du mouvement pour $\xmu$ est l'équation d'onde:
$$\left( \frac{\partial^2}{\partial  \tau^2}-\frac{\partial^2}{\partial  \sigma^2}\right) \xmu=0$$
On peut donc développer $\xmu$ en somme d'exponentielles complexes en fonction de $\tau$ et de cosinus en fonction de $\sigma$ (pas de sinus de $\sigma$ à cause des contraintes au bord):
\begin{equation}
\xmu(\sigma,\tau)= q_0^\mu + \sqrt{2}a_0^\mu \tau + \sqrt{2}i\sum_{n=-\infty;n\neq 0}^{+\infty}\frac{a_n^\mu}{n}\cos(n\sigma)e^{-in\tau}
\end{equation}
$q_0^\mu$ et $a_n^\mu$ sont des constantes indépendantes de $\sigma$ et de $\tau$
\subsubsection{Contraintes sur les modes normaux}
$\xmu$ est réel donc:
$q_0^\mu=q_0^{\mu*}$, $a_0^\mu=a_0^{\mu*}$ et $a_{-n}^\mu=a_{n}^{\mu*}$
\begin{flalign*}
x_+=2\CP_+\tau=q_0^+ + \sqrt{2}a_0^+ \tau + \sqrt{2}i\sum_{n=-\infty;n\neq 0}^{+\infty}\frac{a_n^+}{n}\cos(n\sigma)e^{-in\tau}
\end{flalign*}
 donc $q_0^+ =0$, $a_0^+ =\sqrt{2}\CP_+$ et $a_n^+ =0$ $\forall n\neq 0$
\begin{flalign*}
x_- &=q_0^- + \sqrt{2}a_0^- \tau + \sqrt{2}i\sum_{n=-\infty;n\neq 0}^{+\infty}\frac{a_n^-}{n}\cos(n\sigma)e^{-in\tau}\\
&= q_-(\tau) + \frac{\pi}{\mathcal{P_+}}\int_0^\pi d\sigma'\left( \frac{\sigma'}{\pi}-\theta(\sigma'-\sigma)\right) \vecx'\vecpt\\
&=q_{0-}+\frac{H}{\CP_+}\tau + \frac{\pi}{\mathcal{P_+}}\int_0^\pi d\sigma'\left( \frac{\sigma'}{\pi}-\theta(\sigma'-\sigma)\right) \vecx'\vecpt
\end{flalign*}
donc $q_0^-=q_{0-}$ et $a_0^-=\frac{H}{\sqrt{2}\CP_+}=\frac{H}{a_0^+}$\\
Pour exprimer les $a_n^-$, il faut travailler les expressions un peu plus (cf. Annexe 1).\\
On trouve que $\forall m\neq 0$:
\begin{equation*}
a_m^-=\frac{1}{2a_0^+}\sum_{k=-\infty}^{+\infty}\vec{a}_{-k}\vec{a}_{m+k}
\end{equation*}
On montre également que:
\begin{equation*}
H=\frac{1}{2}\sum_{n=-\infty}^{+\infty}\vec{a}_{-n}\vec{a}_{n}=\frac{(\vec{a}_0)^2}{2}+\sum_{n=1}^{+\infty}\vec{a}_{-n}\vec{a}_{n}
\end{equation*}
Et par conséquent, $\forall n \in \mathbb{Z}$:
\begin{equation}
a_n^-=\frac{1}{2a_0^+}\sum_{k=-\infty}^{+\infty}\vec{a}_{-k}\vec{a}_{n+k}
\end{equation}
\subsubsection{Crochets de Poisson des modes normaux}
Les variables indépendantes sont $\{a_n^i,q_0^i,a_n^+,q_0^-\}$. Tous les autres modes s'expriment en fonction de ces variables comme vu au point (6.3.1). On définit les modes dépendants:
\begin{equation}
L_n=\frac{1}{2}\sum_{k=-\infty}^{+\infty}\vec{a}_{-k}\vec{a}_{n+k}
\end{equation}
Nous pouvons à présent calculer les crochets de Poisson parmi les variables indépendantes ainsi que parmi les modes dépendants (cf.Annexe 2).
\begin{flalign*}
&\{a_n^i,a_m^j\}=-in\delta^{ij}\delta_{n,-m}\\
&\{a_0^i,a_0^j\}=0\\
&\{q_0^i,a_0^j\}=\sqrt{2}\delta^{ij}\\
&\{q_0^i,q_0^j\}=0\\
&\{q_0^-,a_0^+\}=-\sqrt{2}
\end{flalign*}
Calculons à présent les crochets de Poisson des modes dépendants:
\begin{flalign*}
&\{L_n,L_m\}=i(m-n)L_{m+n}\\
&\{L_n,a^j_m\}=ima^j_{m+n}\\
&\{q_0^i,L_{m}\}=\sqrt{2}a_{m}^i
\end{flalign*}
On a montré que:
\begin{equation*}
H=\frac{(\vec{a}_0)^2}{2}+\sum_{n=1}^{+\infty}\vec{a}_{-n}\vec{a}_{n}
\end{equation*}
Or $\vec{a}_{-n}=\vec{a}_{n}^{*}$
$$\Rightarrow H=\frac{(\vec{a}_0)^2}{2}+\sum_{n=1}^{+\infty}\norm{\vec{a}_{n}}^2$$
D'autre part,
$$\vec{\CP}=\int_0^\pi d\sigma \vec{\pt}$$
or $$\vec{\pt}=\frac{1}{\sqrt{2}\pi}\sum_{k=-\infty}^{+\infty}\vec{a}_k\cos(k\sigma)e^{-ik\tau}$$
Donc
$$\vec{\CP}=\frac{1}{\sqrt{2}\pi}\sum_{k=-\infty}^{+\infty}\vec{a}_k e^{-ik\tau}\int_0^\pi\cos(k\sigma)d\sigma=\frac{1}{\sqrt{2}\pi}\sum_{k=-\infty}^{+\infty}\vec{a}_k e^{-ik\tau}\pi\delta_{0,k}=\frac{\vec{a}_0}{\sqrt{2}}$$
Conclusion: $$H=\vec{\CP}^2+\sum_{n=1}^{+\infty}\norm{\vec{a}_{n}}^2$$
Masse invariante: $$M^2=2\CP_+\CP_- - \vec{\CP}^2=\sum_{n=1}^{+\infty}\norm{\vec{a}_{n}}^2$$
\subsubsection{Expression des moments cinétiques et angulaires}
$$\CP^\mu=\int_0^\pi \pt^\mu d\sigma$$
Or
\begin{equation*}
\xmu(\sigma,\tau)= q_0^\mu + \sqrt{2}a_0^\mu \tau + \sqrt{2}i\sumnotzero{n}\frac{a_n^\mu}{n}\cos(n\sigma)e^{-in\tau}
\end{equation*}
Donc
\begin{flalign*}
\pt^\mu&=\frac{\dot{\xmu}}{2\pi}= \frac{\sqrt{2}a_0^\mu}{2\pi} + \frac{\sqrt{2}}{2\pi}\sum_{n=-\infty;n\neq 0}^{+\infty}a_n^\mu\cos(n\sigma)e^{-in\tau}\\
&=\frac{\sqrt{2}}{2\pi}\left[a_0^\mu+\sumnotzero{n}a_n^\mu\cos(n\sigma)e^{-in\tau}\right]=\frac{\sqrt{2}}{2\pi}\sum_{n=-\infty}^{+\infty}a_n^\mu\cos(n\sigma)e^{-in\tau}
\end{flalign*}
Ainsi,
\begin{flalign}
&\CP^\mu=\frac{1}{\sqrt{2}\pi}\sum_{n=-\infty}^{+\infty}a_n^\mu e^{-in\tau}\int_0^\pi\cos(n\sigma)d\sigma= \frac{a_0^\mu}{\sqrt{2}}\\
& M^{\mu\nu}=\int_0^\pi (\xmu\pt^\nu - x^\nu\pt^\mu)d\sigma = \frac{1}{\sqrt{2}}(q_0^\mu a_0^\nu - q_0^\nu a_0^\mu) + i\sumnotzero{n}\frac{a_n^\mu a_{-n}^\nu }{n}\label{Mmunu}
\end{flalign}
\textbf{Démonstration de \eqref{Mmunu}:}
\begin{flalign*}
M^{\mu\nu}&=\int_0^\pi \left(q_0^\mu + \sqrt{2}a_0^\mu \tau + \sqrt{2}i\sumnotzero{n}\frac{a_n^\mu}{n}\cos(n\sigma)e^{-in\tau}\right)\left(\frac{\sqrt{2}}{2\pi}\sum_{k=-\infty}^{+\infty}a_k^\nu\cos(k\sigma)e^{-ik\tau}\right)d\sigma\\
& - \int_0^\pi \left(q_0^\nu + \sqrt{2}a_0^\nu \tau + \sqrt{2}i\sumnotzero{n}\frac{a_n^\nu}{n}\cos(n\sigma)e^{-in\tau}\right)\left(\frac{\sqrt{2}}{2\pi}\sum_{k=-\infty}^{+\infty}a_k^\mu\cos(k\sigma)e^{-ik\tau}\right)d\sigma\\
&=\left(q_0^\mu + \sqrt{2}a_0^\mu\right)\frac{\sqrt{2}}{2\pi}\sum_{k=-\infty}^{+\infty}a_k^\nu e^{-ik\tau}\int_0^\pi\cos(k\sigma)d\sigma + \frac{i}{\pi}\sumnotzero{n}\sum_{k=-\infty}^{+\infty}\frac{a_n^\mu}{n}a_k^\nu e^{-i(n+k)\tau}\int_0^\pi\cos(n\sigma)\cos(k\sigma)d\sigma \\
& - \left(q_0^\nu + \sqrt{2}a_0^\nu\right)\frac{\sqrt{2}}{2\pi}\sum_{k=-\infty}^{+\infty}a_k^\mu e^{-ik\tau}\int_0^\pi\cos(k\sigma)d\sigma + \frac{i}{\pi}\sumnotzero{n}\sum_{k=-\infty}^{+\infty}\frac{a_n^\nu}{n}a_k^\mu e^{-i(n+k)\tau}\int_0^\pi\cos(n\sigma)\cos(k\sigma)d\sigma\\
&=\left(q_0^\mu + \sqrt{2}a_0^\mu\right)\frac{\sqrt{2}}{2}a_0^\nu + \frac{i}{2}\sumnotzero{n}\sum_{k=-\infty}^{+\infty}\frac{a_n^\mu}{n}a_k^\nu e^{-i(n+k)\tau}(\delta_{n,k} + \delta_{n,-k}) \\
& - \left(q_0^\nu + \sqrt{2}a_0^\nu\right)\frac{\sqrt{2}}{2}a_0^\mu + \frac{i}{2}\sumnotzero{n}\sum_{k=-\infty}^{+\infty}\frac{a_n^\nu}{n}a_k^\mu e^{-i(n+k)\tau}(\delta_{n,k} + \delta_{n,-k})\\
\end{flalign*}
\begin{flalign*}
\rightarrow\Mup{\mu\nu}&=\frac{1}{\sqrt{2}}\left(q_0^\mu a_0^\nu - q_0^\nu a_0^\mu\right) + \frac{i}{2}\sumnotzero{n}\left[\frac{a_n^\mu a_n^\nu}{n}e^{-2in\tau} + \frac{a_n^\mu a_{-n}^\nu}{n} - \frac{a_n^\nu a_n^\mu}{n}e^{-2in\tau} - \frac{a_n^\nu a_{-n}^\mu}{n}\right]\\
&=\frac{1}{\sqrt{2}}\left(q_0^\mu a_0^\nu - q_0^\nu a_0^\mu\right) + i\sumnotzero{n}\frac{a_n^\mu a_{-n}^\nu}{n} \square
\end{flalign*}
A partir de cette dernière expression, on peut montrer que les $\Mup{\mu\nu}$  forment une base des représentations du groupe de Lorentz car elles obéissent à la relation:
\begin{equation}
\{M^{\mu\nu},M^{\rho\sigma}\}=g^{\mu\rho}M^{\nu\sigma}- g^{\nu\rho}M^{\mu\sigma} -g^{\mu\sigma}M^{\nu\rho} + g^{\nu\sigma}M^{\mu\rho}\label{poissonbracket}
\end{equation}
où $g^{\mu\nu}$ est la métrique sur notre surface.
\part*{Conclusion}
Ceci achève l'étude classique de la corde relativiste. Le passage au formalisme quantique se fait par le principe de correspondance
$$i\{Crochets de Poisson\} \rightarrow \left[Commutateur\right]$$
Grâce à l'étude poussée des crochets de Poisson réalisée dans la partie classique, on obtient directement les commutateurs des modes normaux pour le cas quantique. La plus grande difficulté réside ensuite à tenir compte de l'ordre des termes dans les modes dépendants. C'est enfin l'analogue du calcul de l'équation \eqref{poissonbracket} qui impose que $D=26$ pour que la théorie soit bien covariante.
\part*{Références}
\begin{enumerate}
\item\label{article} GODDARD P., GOLDSTONE J., REBBI C. and THORN C.B., \textit{Quantum Dynamics of a Massless Relativistic String} (1973), Nuclear Physics B56 , North-Holland Publishing Company
\item\label{zwiebach} ZWIEBACH Barton, \textit{A First Course In String Theory} (2009), Cambridge University Presse
\item\label{QFT} PESKIN E., SCHROEDER D., \textit{An Introduction to Quantum Field Theory} (1995), Perseus Books Publishing
\end{enumerate}
\newpage
\part*{Annexes}
\section*{Annexe 1: Calcul des contraintes sur les $a_n$}
\paragraph{Calcul des coefficients $a_n^-$ par analyse de Fourier pour $n\neq 0$:}
\begin{flalign*}
 f(\sigma,\tau)&\equiv\sqrt{2}i\sum_{n=-\infty;n\neq 0}^{+\infty}\frac{a_n^-}{n}\cos(n\sigma)e^{-in\tau}\\
 &=\frac{\sqrt{2}\pi}{a_0^+}\int_0^\pi d\sigma'\left( \frac{\sigma'}{\pi}-\theta(\sigma'-\sigma)\right) \vecx'\vecpt
\end{flalign*}
Par la première égalité:
\begin{flalign}
\int_0^\pi f(\sigma,\tau)\cos(m\sigma)d\sigma&=\int_0^\pi\left( \sqrt{2}i\sumnotzero{n}\frac{a_n^-}{n}\cos(n\sigma)e^{-in\tau}\right) \cos(m\sigma)d\sigma\\
&=\sqrt{2}i\sumnotzero{n}\frac{a_n^-}{n}e^{-in\tau}\int_0^\pi\cos(n\sigma)\cos(m\sigma)d\sigma\\
&=\sqrt{2}i\sumnotzero{n}\frac{a_n^-}{n}e^{-in\tau}\frac{\pi}{2}\left( \delta_{n,-m}+\delta_{n,m}\right)\label{annexe1}\\ 
&=\frac{i\pi}{\sqrt{2}}\left( \frac{a_{-m}^-}{-m}e^{im\tau}+\frac{a_m^-}{m}e^{-im\tau}\right) \label{annexe2}
\end{flalign}
Le passage de \eqref{annexe1} à \eqref{annexe2} est valable seulement dans le cas $m\neq 0$. 
Si $m=0$, on voit que par la contrainte $n\neq0$ dans la somme, l'intégrale donne 0 (le cas $m=0$ sera traité à part). Nous pouvons donc supposer $m\neq 0$ pour la suite.\\
Par la seconde égalité:
\begin{equation*}
\int_0^\pi f(\sigma,\tau)\cos(m\sigma)d\sigma=\int_0^\pi\left[\frac{\sqrt{2}\pi}{a_0^+}\int_0^\pi d\sigma'\left( \frac{\sigma'}{\pi}-\theta(\sigma'-\sigma)\right) \vecx'\vecpt\right]\cos(m\sigma)d\sigma 
\end{equation*}
Or,
\begin{flalign*}
\vec{x}^{ '} =\frac{\partial}{\partial \sigma'}\left( \vec{q}_0 + \sqrt{2}\vec{a}_0 \tau + \sqrt{2}i\sumnotzero{n}\frac{\vec{a}_n}{n}\cos(n\sigma')e^{-in\tau}\right)
&=\sqrt{2}i\sumnotzero{n}\frac{\vec{a}_n}{n}(-n\sin(n\sigma'))e^{-in\tau}\\
\vec{\pt} =\frac{1}{2\pi}\frac{\partial}{\partial \tau}\left( \vec{q}_0 + \sqrt{2}\vec{a}_0 \tau + \sqrt{2}i\sumnotzero{k}\frac{\vec{a}_k}{k}\cos(k\sigma')e^{-ik\tau}\right)
&=\frac{i}{\sqrt{2}\pi}\sum_{k=-\infty}^{+\infty}\frac{\vec{a}_k}{k}\cos(k\sigma')(-ike^{-ik\tau})\\
\Rightarrow \vec{x}^{ '}\vec{\pt}=\frac{-i}{\pi}\sumnotzero{n}\sum_{k=-\infty}^{+\infty}\vec{a}_n\vec{a}_k&\sin(n\sigma')\cos(k\sigma')e^{-in\tau}e^{-ik\tau}
\end{flalign*}
Ainsi,
\begin{flalign*}
f(\sigma,\tau)&=\frac{-\sqrt{2}i}{a_0^+}\sumnotzero{n}\sum_{k=-\infty}^{+\infty}\vec{a}_n\vec{a}_ke^{-i(n+k)\tau}\left[\int_0^\pi d\sigma'\left( \frac{\sigma'}{\pi}-\theta(\sigma'-\sigma)\right) \sin(n\sigma')\cos(k\sigma')\right]\\
&=\frac{-\sqrt{2}i}{a_0^+}\sumnotzero{n}\vec{a}_n\vec{a}_0e^{-in\tau}\left[\int_0^\pi d\sigma'\left( \frac{\sigma'}{\pi}-\theta(\sigma'-\sigma)\right) \sin(n\sigma')\right]\\
&+\frac{-\sqrt{2}i}{a_0^+}\sumnotzero{n}\sumnotzero{k}\vec{a}_n\vec{a}_ke^{-i(n+k)\tau}\left[\int_0^\pi d\sigma'\left( \frac{\sigma'}{\pi}-\theta(\sigma'-\sigma)\right) \sin(n\sigma')\cos(k\sigma')\right]
\end{flalign*}
Où on a séparé le cas $k=0$ du reste. Ceci implique que le second terme est symétrique en $k$ et $n$ ($\equiv$ on peut interchanger $k$ et $n$ et conserver la même expression).\\
Cependant,
\begin{flalign*}
\int_0^\pi d\sigma'\left( \frac{\sigma'}{\pi}-\theta(\sigma'-\sigma)\right) \sin(n\sigma')\cos(k\sigma')&=\frac{1}{2\pi}\int_0^\pi \sigma'\left[ \sin((n+k)\sigma')+\sin((n-k)\sigma')\right]d\sigma'\\ &-\frac{1}{2}\int_{\sigma}^{\pi}\left[ \sin((n+k)\sigma')+\sin((n-k)\sigma')\right]d\sigma'
\end{flalign*}
Or $\sin((n-k)\sigma')$ est antisymétrique par rapport à l'échange de $k$ et $n$ $\Rightarrow$ les termes qui contiennent $\sin((n-k)\sigma')$ seront nuls. On peut donc ré-écrire:
\begin{flalign*}
f(\sigma,\tau)&=\frac{-\sqrt{2}i}{a_0^+}\sum_{n=-\infty;n\neq 0}^{+\infty}\vec{a}_n\vec{a}_0e^{-in\tau}\left[\frac{1}{\pi}\int_0^\pi \sigma'\sin(n\sigma')d\sigma'-\int_\sigma^\pi \sin(n\sigma')d\sigma'\right]\\
&+\frac{-\sqrt{2}i}{a_0^+}\sum_{n=-\infty;n\neq 0}^{+\infty}\sum_{k=-\infty;k\neq 0}^{+\infty}\vec{a}_n\vec{a}_ke^{-i(n+k)\tau}\left[\frac{1}{2\pi}\int_0^\pi \sigma' \sin((n+k)\sigma')d\sigma'-\frac{1}{2}\int_{\sigma}^{\pi} \sin((n+k)\sigma')d\sigma'\right]
\end{flalign*}
On doit maintenant traiter les intégrales des sinus:\\
Si $a\neq$0, alors:
\begin{flalign*}
&\int_0^\pi x\sin(ax) dx=\left[ x\frac{\cos(ax)}{-a}\right]_0^\pi - \int_0^\pi \frac{\cos(ax)}{-a}dx=\left( -\frac{\pi}{a}\cos(a\pi)\right) +\left[\frac{\sin(ax)}{a^2}\right]_0^\pi=-\frac{\pi}{a}\cos(a\pi)\\
&\int_\sigma^\pi \sin(ax) dx = \left[\frac{\cos(ax)}{-a}\right]_\sigma^\pi=\frac{\cos(a\sigma)-\cos(a\pi)}{a}
\end{flalign*}
Si $a=0$, alors $\sin(ax)=0$ $\forall x$\\
On introduit le symbole suivant:
\begin{equation*}
\epsilon_{a}=\begin{cases} 0,& \mbox{si } a=0\\1, & \mbox{si } a\neq 0 \end{cases}
\end{equation*}
Alors:
\begin{equation*}
\frac{1}{\pi}\int_0^\pi x\sin(ax) dx-\int_\sigma^\pi \sin(ax) dx=\epsilon_{a}*\left(-\frac{\cos(a\sigma)}{a}\right)
\end{equation*}
Donc
\begin{flalign*}
f(\sigma,\tau)&=\frac{\sqrt{2}i}{a_0^+}\sumnotzero{n}\left[ \frac{\vec{a}_n\vec{a}_0}{n}\cos(n\sigma)e^{-in\tau}+\sumnotzero{k}\epsilon_{n+k}\frac{\vec{a}_n\vec{a}_k}{2(n+k)}\cos((n+k)\sigma)e^{-i(n+k)\tau}\right]\\
&=\frac{\sqrt{2}i}{a_0^+}\sum_{n=-\infty}^{+\infty}\sum_{\substack{k=-\infty\\k\neq n}}^{+\infty}\frac{\vec{a}_n\vec{a}_k}{2(n+k)}\cos((n+k)\sigma)e^{-i(n+k)\tau}
\end{flalign*}
On peut finalement intégrer $f(\sigma,\tau)$:
\begin{flalign*}
\int_0^\pi f(\sigma,\tau)\cos(m\sigma)d\sigma&=\int_0^\pi\left[  \frac{\sqrt{2}i}{a_0^+}\sum_{n=-\infty}^{+\infty}\sum_{\substack{k=-\infty\\k\neq n}}^{+\infty}\frac{\vec{a}_n\vec{a}_k}{2(n+k)}\cos((n+k)\sigma)e^{-i(n+k)\tau}\right] \cos(m\sigma)d\sigma\\
&=\frac{\sqrt{2}i}{a_0^+}\sum_{n=-\infty}^{+\infty}\sum_{\substack{k=-\infty\\k\neq n}}^{+\infty}\frac{\vec{a}_n\vec{a}_k}{2(n+k)}e^{-i(n+k)\tau}\int_0^\pi\cos((n+k)\sigma)\cos(m\sigma)d\sigma
\end{flalign*}
\begin{equation*}
\int_0^\pi\cos(ax)\cos(bx)dx=\frac{1}{2}\int_0^\pi\left( \cos((a+b)x)+\cos((a-b)x\right)dx=\frac{\pi}{2}\left( \delta_{a,-b}+\delta_{a,b}\right)
\end{equation*}
Car
$$\int_0^\pi\cos(ax)dx=\pi\delta_{a,0}$$
Donc
\begin{flalign*}
\int_0^\pi f(\sigma,\tau)\cos(m\sigma)d\sigma= \frac{\pi i}{2\sqrt{2}a_0^+}\sum_{n=-\infty}^{+\infty}\sum_{\substack{k=-\infty\\k\neq n}}^{+\infty}\frac{\vec{a}_n\vec{a}_k}{n+k}e^{-i(n+k)\tau}\left( \delta_{n+k,-m}+\delta_{n+k,m}\right) 
\end{flalign*}
Comme $m\neq 0$,
\begin{flalign*}
\int_0^\pi f(\sigma,\tau)\cos(m\sigma)d\sigma&= \frac{\pi i}{2\sqrt{2}a_0^+}\sum_{n=-\infty}^{+\infty}\left[ \frac{\vec{a}_n\vec{a}_{-m-n}}{-m}e^{im\tau}+\frac{\vec{a}_n\vec{a}_{m-n}}{m}e^{-im\tau}\right]\\
&=\frac{i\pi}{\sqrt{2}}\left( \frac{e^{im\tau}}{-m}\frac{1}{2a_0^+}\sum_{k=-\infty}^{+\infty} \vec{a}_{-k}\vec{a}_{-m+k}+\frac{e^{-im\tau}}{m}\frac{1}{2a_0^+}\sum_{k=-\infty}^{+\infty}\vec{a}_{-k}\vec{a}_{m+k}\right)  
\end{flalign*}
En comparant cette dernière égalité à l'expression \eqref{annexe2}, on voit directement que:
\begin{equation}
a_m^-=\frac{1}{2a_0^+}\sum_{k=-\infty}^{+\infty}\vec{a}_{-k}\vec{a}_{m+k}
\end{equation}
\paragraph{Calcul de $a_0^-$:}
On sait que $a_0^-=\frac{H}{a_0^+}$
Nous devons donc exprimer $H$ en fonction des modes normaux.
$$H=\pi\int_0^\pi\left( \vecpt^2 +\frac{\vecx^{'2}}{(2\pi)^2}\right)d\sigma$$
Or
\begin{equation*}
\left\lbrace
\begin{aligned}
\vec{x}^{ '} =\sqrt{2}i\sumnotzero{n}\frac{\vec{a}_n}{n}(-n\sin(n\sigma))e^{-in\tau}\\
\vec{\pt}=\frac{i}{\sqrt{2}\pi}\sum_{k=-\infty}^{+\infty}\frac{\vec{a}_k}{k}\cos(k\sigma)(-ike^{-ik\tau})
\end{aligned}
\right.
\end{equation*}
\begin{equation*}
\Rightarrow
\left\lbrace
\begin{aligned} \frac{(\vec{x}^{'})^2}{(2\pi)^2}=\frac{-2}{(2\pi)^2}\sumnotzero{k}\sumnotzero{n}\vec{a}_k\vec{a}_ne^{-i(n+k)\tau}\sin(k\sigma)\sin(n\sigma)\\
(\vec{\pt})^2=\frac{1}{2\pi^2}\sum_{n=-\infty}^{+\infty}\sum_{k=-\infty}^{+\infty}\vec{a}_n\vec{a}_ke^{-i(n+k)\tau}\cos(n\sigma)\cos(k\sigma)
\end{aligned}
\right.
\end{equation*}
Donc
\begin{flalign*}
H&=\frac{1}{2\pi}\int_0^\pi d\sigma\left[ \sumnotzero{k}\sumnotzero{n}\vec{a}_k\vec{a}_ne^{-i(n+k)\tau}(\cos(n\sigma)\cos(k\sigma)-\sin(n\sigma)\sin(k\sigma))\right]+\frac{1}{2\pi}\int_0^\pi d\sigma(\vec{a}_0)^2 \\
&=\frac{1}{2\pi}\left[ \sumnotzero{k}\sumnotzero{n}\vec{a}_k\vec{a}_ne^{-i(n+k)\tau}\int_0^\pi  \cos((n+k)\sigma)d\sigma\right] +\frac{(\vec{a}_0)^2}{2}\\
&=\frac{1}{2}\left[ \sumnotzero{k}\sumnotzero{n}\vec{a}_k\vec{a}_ne^{-i(n+k)\tau}\delta_{n,-k}\right] +\frac{(\vec{a}_0)^2}{2}=\left( \sumnotzero{n}\frac{\vec{a}_{-n}\vec{a}_n}{2}\right)  +\frac{(\vec{a}_0)^2}{2}
\end{flalign*}
\begin{equation}
\Rightarrow H=\frac{1}{2}\sum_{n=-\infty}^{+\infty}\vec{a}_{-n}\vec{a}_{n}=\frac{(\vec{a}_0)^2}{2}+\sum_{n=1}^{+\infty}\vec{a}_{-n}\vec{a}_{n}
\end{equation}
\paragraph{Conclusion:}
$\forall n \in \mathbb{Z}:$
\begin{equation}
a_n^-=\frac{1}{2a_0^+}\sum_{k=-\infty}^{+\infty}\vec{a}_{-k}\vec{a}_{n+k}
\end{equation}
\newpage
\section*{Annexe 2: Calcul des crochets de Poisson des modes normaux}
Nous allons calculer les crochets de Poissons parmi les variables indépendantes et les modes dépendants.\\
Lors du calcul des contraintes sur les $a_n^-$, on a montré que:
\begin{equation}
\left\lbrace
\begin{aligned}
&x^i= q_0^i + \sqrt{2}a_0^i \tau + \sqrt{2}i\sumnotzero{n}\frac{a_n^i}{n}\cos(n\sigma)e^{-in\tau}\\
&\pt^i=\sum_{k=-\infty}^{+\infty}\frac{a_k^i}{\sqrt{2}\pi}\cos(k\sigma)e^{-ik\tau}
\end{aligned}
\right.
\end{equation}
Donc si on veut isoler les $a_n^i$ en fonction de $x^i$ et $\pt^i$:
\begin{flalign*}
\int_0^\pi \pt^i \cos(m\sigma)d\sigma&=\sum_{k=-\infty}^{+\infty}\frac{a_k^i}{\sqrt{2}\pi}e^{-ik\tau}\int_0^\pi\cos(k\sigma)cos(m\sigma)d\sigma=\sum_{k=-\infty}^{+\infty}\frac{a_k^i}{\sqrt{2}\pi}e^{-ik\tau}\frac{\pi}{2}\left( \delta_{k,m}+\delta_{k,-m}\right) \\
&=\frac{1}{2\sqrt{2}}\left(a_m^ie^{-im\tau}+a_{-m}^ie^{im\tau}\right) 
\end{flalign*}
\begin{flalign*}
\int_0^\pi x^i \cos(m\sigma)d\sigma &=  (q_0^i + \sqrt{2}a_0^i \tau)\int_0^\pi\cos(m\sigma)d\sigma + \sqrt{2}i\sumnotzero{n}\frac{a_n^i}{n}e^{-in\tau}\int_0^\pi\cos(n\sigma)\cos(m\sigma)d\sigma\\
&=\pi\delta_{m,0}(q_0^i + \sqrt{2}a_0^i \tau) + \epsilon_{m,0}\frac{\pi\sqrt{2}i}{2}\left( \frac{a_m^i}{m}e^{-im\tau} + \frac{a_{-m}^i}{-m}e^{im\tau})\right) 
\end{flalign*}
Donc $\forall m\neq 0$:
$$a^i_m=e^{im\tau}\left( \frac{m}{\sqrt{2}i\pi}\int_0^\pi x^i\cos(m\sigma)d\sigma + \sqrt{2}\int_0^\pi \pt^i \cos(m\sigma)d\sigma\right) $$
Et pour $m=0$:
$$a_0^i=\sqrt{2}\int_0^\pi\pt^i d\sigma$$
$$\int_0^\pi x^id\sigma=\pi(q_0^i+\sqrt{2}a_0^i\tau)$$
$$\Rightarrow q_0^i=\int_0^\pi(\frac{x^i}{\pi}- 2\tau \pt^i)d\sigma$$
\textbf{Crochets de Poisson}:\\
\begin{flalign*}
\{a_n^i,a_m^j\}&=e^{in\tau}e^{jm\tau}\left\{\int_0^\pi\left( \frac{nx^i}{\sqrt{2}i\pi}+\sqrt{2}\pt^i\right) \cos(n\sigma)d\sigma,\int_0^\pi\left(\frac{mx^j}{\sqrt{2}i\pi}+\sqrt{2}\pt^j\right)\cos(m\sigma')d\sigma' \right\}
\end{flalign*}
Comme $x^i(\sigma)$ et $\pt^i(\sigma)$ sont canoniquement conjugués, seuls les termes croisés seront non nuls:
\begin{flalign*}
\{a_n^i,a_m^j\}&=\frac{-i}{\pi}e^{in\tau}e^{jm\tau}\int_0^\pi d\sigma\int_0^\pi d\sigma'\cos(n\sigma)\cos(m\sigma')\left(\{nx^i(\sigma),\pt^j(\sigma')\} + \{ \pt^i(\sigma) ,mx^j(\sigma')\}\right)\\
&= \frac{-i}{\pi}e^{in\tau}e^{jm\tau}\int_0^\pi d\sigma\int_0^\pi d\sigma'\cos(n\sigma)\cos(m\sigma')\left(n\delta^{ij}\delta(\sigma-\sigma') - m\delta^{ij}\delta(\sigma-\sigma')\right)\\
&= \frac{-i}{\pi}e^{in\tau}e^{im\tau}\delta^{ij}\int_0^\pi d\sigma\cos(n\sigma)\cos(m\sigma)(n - m)\\
&= \frac{i(m-n)}{\pi}e^{in\tau}e^{im\tau}\delta^{ij}\left[\frac{\pi}{2}(\delta_{n,m}+\delta_{n,-m})\right]\\
&=\left[\frac{i(m-m)}{2}e^{im\tau}e^{im\tau}\delta^{ij}\delta_{n,m}+\frac{i(m-(-m))}{2}e^{-im\tau}e^{im\tau}\delta^{ij}\delta_{n,-m}\right]\\
&=im\delta^{ij}\delta_{n,-m}=-in\delta^{ij}\delta_{n,-m}\\
\{a_0^i,a_0^j\}&=2\int_0^\pi d\sigma\int_0^\pi d\sigma'\{\pt^i(\sigma),\pt^j(\sigma')\}=0\\
\{q_0^i,a_0^j\}&=\left\{\frac{1}{\pi}\int_0^\pi  (x^i(\sigma)-2\tau\pt^i(\sigma))d\sigma,\sqrt{2}\int_0^\pi \pt^j(\sigma')d\sigma'\right\}=\frac{\sqrt{2}}{\pi}\int_0^\pi d\sigma \int_0^\pi d\sigma' \{x^i(\sigma),\pt^j(\sigma')\}\\
&=\frac{\sqrt{2}}{\pi}\delta^{ij}\int_0^\pi d\sigma \int_0^\pi d\sigma'\delta(\sigma-\sigma')=\sqrt{2}\delta^{ij}\\
\{q_0^i,q_0^j\}&=\left\{\frac{1}{\pi}\int_0^\pi  (x^i(\sigma)-2\tau\pt^i(\sigma))d\sigma,\frac{1}{\pi}\int_0^\pi  (x^j(\sigma')-2\tau\pt^j(\sigma'))d\sigma'\right\}\\
&=\frac{-2\tau}{\pi}\int_0^\pi d\sigma\int_0^\pi d\sigma'\left(\{  x^i(\sigma),\pt^j(\sigma')\} +\{\pt^i(\sigma),x^j(\sigma')\}\right)=0\\
\{q_0^-,a_0^+\}&=\{q_{0-},\sqrt{2}\CP_+\}=-\sqrt{2}
\end{flalign*}
Calculons à présent les crochets de Poisson des modes dépendants:
\begin{flalign*}
\{L_n,L_m\}=\frac{1}{4}\sum_{k=-\infty}^{k=+\infty}\sum_{l=-\infty}^{l=+\infty}&\{\vec{a}_{-k} \cdot \vec{a}_{n+k},\vec{a}_{-l} \cdot \vec{a}_{m+l}\}=\frac{1}{4}\sum_{k=-\infty}^{k=+\infty}\sum_{l=-\infty}^{l=+\infty}\sum_{i=1}^{D-2}\sum_{j=1}^{D-2}\{a^i_{-k}a^i_{n+k},a^j_{-l}a^j_{m+l}\}\\
\{a^i_{-k} \cdot a^i_{n+k},a^j_{-l} \cdot a^j_{m+l}\}=& a^i_{-k}\{a^i_{n+k},a^j_{-l}\}a^j_{m+l}+a^i_{-k}\{a^i_{n+k},a^j_{m+l}\}a^j_{-l}\\
+& a^i_{n+k}\{a^i_{-k},a^j_{-l}\}a^j_{m+l}+a^i_{n+k}\{a^i_{-k},a^j_{m+l}\}a^j_{-l}\\
=& a^i_{-k}(-i(n+k)\delta_{n+k,l}\delta^{ij})a^j_{m+l}+a^i_{-k}(-i(n+k)\delta_{n+k,-m-l}\delta^{ij})a^j_{-l}\\
+& a^i_{n+k}(-i(-k)\delta_{-k,l}\delta^{ij})a^j_{m+l}+a^i_{n+k}(-i(-k)\delta_{-k,-m-l}\delta^{ij})a^j_{-l}\\
\end{flalign*}
Donc
\begin{flalign*}
\{L_n,L_m\}=\frac{1}{4}\sum_{k=-\infty}^{k=+\infty}\sum_{l=-\infty}^{l=+\infty}\sum_{i=1}^{D-2}\sum_{j=1}^{D-2}& a^i_{-k}(-i(n+k)\delta_{n+k,l}\delta^{ij})a^j_{m+l}+a^i_{-k}(-i(n+k)\delta_{n+k,-m-l}\delta^{ij})a^j_{-l}\\
&+a^i_{n+k}(-i(-k)\delta_{-k,l}\delta^{ij})a^j_{m+l}+a^i_{n+k}(-i(-k)\delta_{-k,-m-l}\delta^{ij})a^j_{-l}
\end{flalign*}
\begin{flalign*}
&=\frac{1}{4}\sum_{k=-\infty}^{k=+\infty}\sum_{i=1}^{D-2} \left[ -i(n+k)a^i_{-k}a^i_{m+n+k}-i(n+k)a^i_{-k}a^i_{m+n+k}+ika^i_{n+k}a^i_{m-k}+ika^i_{n+k}a^i_{m-k}\right]\\
&=\frac{1}{2}\sum_{k=-\infty}^{k=+\infty}\sum_{i=1}^{D-2} -i(n+k)a^i_{-k}a^i_{m+n+k}+\frac{1}{2}\sum_{k=-\infty}^{k=+\infty}\sum_{i=1}^{D-2}ika^i_{n+k}a^i_{m-k}\\
&=\frac{1}{2}\sum_{k=-\infty}^{k=+\infty}\sum_{i=1}^{D-2} -i(n+k)a^i_{-k}a^i_{m+n+k}+\frac{1}{2}\sum_{k'=-\infty}^{k=+\infty}\sum_{i=1}^{D-2}i(k'+m)a^i_{n+m+k'}a^i_{-k'}\\
&=\frac{1}{2}\sum_{k=-\infty}^{k=+\infty}\sum_{i=1}^{D-2} i(m-n)a^i_{-k}a^i_{m+n+k}=i(m-n)L_{m+n}
\end{flalign*}
\begin{flalign*}
\{L_n,a^j_m\}&=\frac{1}{2}\sum_{k=-\infty}^{+\infty}\sum_{i=1}^{D-2}\{a^i_{-k}a^i_{n+k},a^j_m\}=\frac{1}{2}\sum_{k=-\infty}^{+\infty}\sum_{i=1}^{D-2}\left[ a^i_{-k}\{a^i_{n+k},a^j_m\}+\{a^i_{-k},a^j_m\}a^i_{n+k}\right]\\
&=\frac{1}{2}\sum_{k=-\infty}^{+\infty}\sum_{i=1}^{D-2}\left[ a^i_{-k}(-i(n+k)\delta_{n+k,-m}\delta_{i,j})+(-i(-k)\delta_{-k,-m}\delta_{i,j}a^i_{n+k}\right]\\
&=\frac{1}{2}\left[ima^j_{m+n}+ima^j_{n+m}\right]=ima^j_{m+n}
\end{flalign*}
\begin{flalign*}
\{q_0^i,L_{m}\}&=\frac{1}{2}\sum_{k=-\infty}^{+\infty}\sum_{j=1}^{D-2}\{q_0^i,a_{-k}^j \cdot a_{m+k}^{j}\}=\frac{1}{2}\sum_{k=-\infty}^{+\infty}\sum_{j=1}^{D-2}\left[\{q_0^i,a_{-k}^j\}a_{m+k}^{j}+a_{-k}^j\{q_0^i,a_{m+k}^{j}\}\right]\\
&=\frac{1}{2}\sum_{k=-\infty}^{+\infty}\sum_{j=1}^{D-2}\left[\sqrt{2}\delta_{0,-k}\delta^{ij}a_{n+k}^{j}+a_{-k}^j\sqrt{2}\delta_{0,m+k}\delta^{ij}\right]=\sqrt{2}a_{m}^i
\end{flalign*}


\end{document}