% !TeX spellcheck = en_US
\documentclass[a4paper,12pt]{article}
\def\xmu{x^\mu}
\def\pt{P_{\tau}}
\newcommand{\Mup}[1]{M^{#1}}
\newcommand{\gup}[1]{g^{#1}}
\newcommand{\norm}[1]{\left\lVert#1\right\rVert}
\usepackage[utf8]{inputenc}
\usepackage{amsmath}
\usepackage{amssymb}
\usepackage[a4paper]{geometry}
\geometry{hscale=0.80,vscale=0.80,centering}
\title{Calcul des crochets de Poisson entre les $\Mup{\mu\nu}$}


\begin{document}
\maketitle
On a montré que les $\Mup{\mu\nu}$ s'écrivaient en fonction des modes normaux comme:
\begin{equation}
\Mup{\mu\nu}=\int_0^\pi (\xmu\pt^\nu - x^\nu\pt^\mu)d\sigma = \frac{1}{\sqrt{2}}(q_0^\mu a_0^\nu - q_0^\nu a_0^\mu) + i\sum_{n=-\infty; n\neq 0}^{n=+\infty}\frac{a_n^\mu a_{-n}^\nu }{n}
\end{equation}
On va calculer les crochets de Poisson de ces $\Mup{\mu\nu}$ et montrer qu'ils  forment une base des représentations du groupe de Lorentz:
\begin{equation}
\{\Mup{\mu\nu},\Mup{\rho\sigma}\}=\gup{\mu\rho}\Mup{\nu\sigma}- \gup{\nu\rho}\Mup{\mu\sigma} -\gup{\mu\sigma}\Mup{\nu\rho} + \gup{\nu\sigma}\Mup{\mu\rho}
\end{equation}
où $\gup{\mu\nu}$ est la métrique sur notre surface.
On doit commencer par calculer notre métrique dans les nouvelles coordonnées. En effet, on a fait un changement de variables:
$r_\pm=\frac{1}{\sqrt2}(r^0\pm r^{D-1})$ donc on doit réexprimer la métrique $\gup{\mu\nu}$ dans ces nouvelles variables.
\begin{flalign*}
\forall i,j \in \{1,D-2\}: &\gup{ij}=\delta^{ij}\\
&\gup{00}=-1 \\
&\gup{D-1 D-1}=1\\
&\gup{0i}=\gup{D-1 i}=\gup{0 D-1}=0
\end{flalign*}
et $$\gup{\mu\nu}=\gup{\nu\mu}$$
Or 
\begin{flalign*}
& ds^2= -dx_0^2 + dx_{D-1}^2 + d\vec{x}^2\\
& dx_0= \frac{1}{\sqrt{2}}(dx_+ + dx_-)\\
& dx_{D-1}= \frac{1}{\sqrt{2}}(dx_+ - dx_-)\\
\Rightarrow & ds^2= -\frac{1}{2}(dx_+ + dx_-)^2 + \frac{1}{2}(dx_+ - dx_-)^2 + d\vec{x}^2\\
&= -2dx_+ dx_- + d\vec{x}^2
\end{flalign*}
Donc:
\begin{flalign*}
\forall i,j \in \{1,D-2\}:  &\gup{ij}=\delta^{ij} \\
&\gup{+-}=\gup{-+}=-1\\
&\gup{i+}\gup{i-}=\gup{++}=\gup{--}=0
\end{flalign*}
et tout le reste est nul.
On va démontrer l'équation (2) en séparant les indices spatiaux (i,j) des indices (+,-):
\section{Partie spatiale}
On commence par la partie spatiale:
Montrons que
\begin{flalign*}
\{\Mup{ij},\Mup{kl}\}&=\gup{ik}\Mup{jl}- \gup{jk}\Mup{il} -\gup{il}\Mup{jk} + \gup{jl}\Mup{ik}=\delta^{ik}\Mup{jl}- \delta^{jk}\Mup{il} -\delta^{il}\Mup{jk} + \delta^{jl}\Mup{ik}\\
\{\Mup{ij},\Mup{kl}\}&=\left\{\frac{1}{\sqrt{2}}(q_0^i a_0^j - q_0^j a_0^i) + i\sum_{n=-\infty; n\neq 0}^{n=+\infty}\frac{a_n^i a_{-n}^j }{n},\frac{1}{\sqrt{2}}(q_0^k a_0^l - q_0^l a_0^k) + i\sum_{m=-\infty; m\neq 0}^{m=+\infty}\frac{a_m^k a_{-m}^l }{m}\right\}\\
&=\frac{1}{2}\left\{q_0^i a_0^j - q_0^j a_0^i,q_0^k a_0^l - q_0^l a_0^k\right\} -\left\{\sum_{n=-\infty; n\neq 0}^{n=+\infty}\frac{a_n^i a_{-n}^j }{n},\sum_{m=-\infty; m\neq 0}^{m=+\infty}\frac{a_m^k a_{-m}^l }{m}\right\}\\
&+\frac{i}{\sqrt{2}}\left( \left\{q_0^i a_0^j - q_0^j a_0^i,\sum_{m=-\infty; m\neq 0}^{m=+\infty}\frac{a_m^k a_{-m}^l }{m}\right\}+\left\{\sum_{n=-\infty; n\neq 0}^{n=+\infty}\frac{a_n^i a_{-n}^j }{n},q_0^k a_0^l - q_0^l a_0^k\right\}\right) 
\end{flalign*}
Traitons cette somme terme à terme:\\
Dans le premier terme, seuls les crochets de Poissons mixtes seront non nuls car $$\{a_0^i,a_0^j\}=\{q_0^i,q_0^j\}=0$$
On utilise que: $$\{q_0^i,a_0^j\}=\sqrt{2}\delta^{ij}$$
Donc
\begin{flalign*}
\frac{1}{2}\{q_0^i a_0^j - q_0^j a_0^i,q_0^k a_0^l - q_0^l a_0^k\}=& \frac{1}{2}\left[ q_0^i\{a_0^j,q_0^k\}a_0^l + a_0^j\{q_0^i,a_0^l\}q_0^k - q_0^i\{a_0^j,q_0^l\}a_0^k - a_0^j\{q_0^i,a_0^k\}q_0^l\right] \\
+&\frac{1}{2} \left[ - q_0^j\{a_0^i,q_0^k\}a_0^l - a_0^i\{q_0^j,a_0^l\}q_0^k + q_0^j\{a_0^i,q_0^l\}a_0^k + a_0^i\{q_0^j,a_0^k\}q_0^l\right] 
\end{flalign*}
\begin{flalign*}
&=\frac{1}{\sqrt{2}}\left[ -q_0^i a_0^l\delta^{jk} + a_0^j q_0^k \delta^{il} + q_0^i a_0^k\delta^{jl} - a_0^j q_0^l\delta^{ik}+ q_0^j a_0^l\delta^{ik} - a_0^i q_0^k\delta^{jl} - q_0^j a_0^k\delta^{il} + a_0^i q_0^l\delta^{jk}\right]\\
&=\frac{1}{\sqrt{2}}\left[\delta^{ik}(q_0^j a_0^l- a_0^j q_0^l) - \delta^{jk}(q_0^i a_0^l-a_0^i q_0^l) - \delta^{il}(q_0^j a_0^k-a_0^j q_0^k) + \delta^{jl}(q_0^i a_0^k- a_0^i q_0^k)  \right]
\end{flalign*}
Deuxième terme:
On utilise : $\{a_n^i,a_m^j\}=-in\delta^{ij}\delta_{n,-m}$
\begin{flalign*}
&-\left\{\sum_{n=-\infty; n\neq 0}^{n=+\infty}\frac{a_n^i a_{-n}^j }{n} ,\sum_{m=-\infty; m\neq 0}^{m=+\infty}\frac{a_m^k a_{-m}^l }{m}\right\}=-\sum_{n=-\infty; n\neq 0}^{n=+\infty}\sum_{m=-\infty; m\neq 0}^{m=+\infty}\frac{1}{nm}\{a_n^i a_{-n}^j,a_m^k a_{-m}^l\}=(**)\\
&\{a_n^i a_{-n}^j,a_m^k a_{-m}^l\}=a_n^i\{ a_{-n}^j,a_m^k \}a_{-m}^l + a_{-n}^j\{a_n^i ,a_m^k \}a_{-m}^l + a_n^i\{ a_{-n}^j, a_{-m}^l\}a_m^k + a_{-n}^j\{a_n^i , a_{-m}^l\}a_m^k\\
&=-i\Big((-n\delta^{jk}\delta_{-n,-m})a_n^i a_{-m}^l + (n\delta^{ik}\delta_{n,-m})a_{-n}^ja_{-m}^l +(-n\delta^{jl}\delta_{-n,m})a_n^i a_m^k + (n\delta^{il}\delta_{n,m})a_{-n}^j a_m^k\Big)\\
&\Rightarrow (**)=+i \sum_{n=-\infty; n\neq 0}^{n=+\infty}\Big(\frac{-1}{n}\delta^{jk}a_n^i a_{-n}^l +\frac{1}{-n}\delta^{ik}a_{-n}^j a_{n}^l +\frac{-1}{-n}\delta^{jl}a_n^i a_{-n}^k + \frac{1}{n}\delta^{il}a_{-n}^j a_n^k\Big)\\
&=-i\sum_{n=-\infty; n\neq 0}^{n=+\infty}\frac{a_n^i a_{-n}^l}{n}\delta^{jk} - i\sum_{n=-\infty; n\neq 0}^{n=+\infty}\frac{a^j_{-n} a^l_n}{n}\delta^{ik}+i\sum_{n=-\infty; n\neq 0}^{n=+\infty}\frac{a^i_n a^k_{-n}}{n}\delta^{jl} + i\sum_{n=-\infty; n\neq 0}^{n=+\infty}\frac{a^j_{-n} a^k_n}{n}\delta^{il}\\
&=\delta^{ik}i\sum_{n=-\infty; n\neq 0}^{n=+\infty}\frac{a^j_{n} a^l_n}{-n}-\delta^{jk}i\sum_{n=-\infty; n\neq 0}^{n=+\infty}\frac{a_n^i a_{-n}^l}{n} - \delta^{il}i\sum_{n=-\infty; n\neq 0}^{n=+\infty}\frac{a^j_{n} a^k_{-n}}{n} +\delta^{jl}i\sum_{n=-\infty; n\neq 0}^{n=+\infty}\frac{a^i_n a^k_{-n}}{n} 
\end{flalign*}
Les troisièmes et quatrièmes termes sont nuls car pour $m\neq 0$, 
$$\{a_0^i,a_{\pm m}^j\}=0$$
$$\{q_0^i,a_{\pm m}^j\}=0$$
Donc:
\begin{flalign*}
\{\Mup{ij},\Mup{kl}\}&=\frac{1}{\sqrt{2}}\left[\delta^{ik}(q_0^j a_0^l- a_0^j q_0^l) - \delta^{jk}(q_0^i a_0^l-a_0^i q_0^l) - \delta^{il}(q_0^j a_0^k-a_0^j q_0^k) + \delta^{jl}(q_0^i a_0^k- a_0^i q_0^k)  \right]\\
&+\delta^{ik}i\sum_{n=-\infty; n\neq 0}^{n=+\infty}\frac{a^j_{n} a^l_n}{-n}-\delta^{jk}i\sum_{n=-\infty; n\neq 0}^{n=+\infty}\frac{a_n^i a_{-n}^l}{n} - \delta^{il}i\sum_{n=-\infty; n\neq 0}^{n=+\infty}\frac{a^j_{n} a^k_{-n}}{n} +\delta^{jl}i\sum_{n=-\infty; n\neq 0}^{n=+\infty}\frac{a^i_n a^k_{-n}}{n}\\
&=\delta^{ik}\left(\frac{1}{\sqrt{2}}(q_0^j a_0^l- a_0^j q_0^l) +i\sum_{n=-\infty; n\neq 0}^{n=+\infty}\frac{a^j_{n} a^l_n}{-n}\right)- \delta^{jk}\left(\frac{1}{\sqrt{2}}(q_0^i a_0^l-a_0^i q_0^l) +i\sum_{n=-\infty; n\neq 0}^{n=+\infty}\frac{a_n^i a_{-n}^l}{n}\right)\\
&- \delta^{il}\left(\frac{1}{\sqrt{2}}(q_0^j a_0^k-a_0^j q_0^k)+i\sum_{n=-\infty; n\neq 0}^{n=+\infty}\frac{a^j_{n} a^k_{-n}}{n}\right) + \delta^{jl}\left(\frac{1}{\sqrt{2}}(q_0^i a_0^k- a_0^i q_0^k) +i\sum_{n=-\infty; n\neq 0}^{n=+\infty}\frac{a^i_n a^k_{-n}}{n}\right)\\
&=\delta^{ik}\Mup{jl}- \delta^{jk}\Mup{il} -\delta^{il}\Mup{jk} + \delta^{jl}\Mup{ik} \square
\end{flalign*}
\section{Partie (+,-)}
Comme $a_n^+ =0$ $\forall n$ et $q_0^+=0$,
On a que: 
\begin{flalign*}
&\Mup{+-}=-\frac{1}{\sqrt{2}}q_0^- a_0^+\\
&\Mup{-+}=\frac{1}{\sqrt{2}}q_0^- a_0^+=-\Mup{+-}\\
&\Mup{++}=0\\
&\Mup{--}=i\sum_{m=-\infty;m\neq 0}^{m=+\infty} \frac{a_m^-a_{-m}^-}{m}=\frac{i}{(a_0^+)^2}\sum_{m=-\infty;m\neq 0}^{m=+\infty}\frac{L_m \cdot L_{-m}}{m}
\end{flalign*}
\end{document}