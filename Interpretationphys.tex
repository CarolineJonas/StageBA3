\documentclass[a4paper,12pt]{article}
\def\xmu{x^\mu}
\usepackage[utf8]{inputenc}
\usepackage{amsmath}
\usepackage[a4paper]{geometry}
\geometry{hscale=0.80,vscale=0.80,centering}
\title{Interprétation physique des équations classiques du mouvement}
\begin{document}
\maketitle

On va voir que cette expression des équations du mouvement correspond à un cas simple que l'on peut comprendre physiquement: 
Soit un référentiel de Lorentz où $$\tau=t$$
Le mouvement en 3 dimensions est alors décrit par le vecteur 
$$\vec{x}=\vec{x}(t,\sigma)$$
Soit $$ds=\vert\frac{\partial x}{\partial \sigma}\vert d\sigma$$ l'élément de longueur le long de la corde.\\
Soit $$v_{\perp}=\frac{\partial \vec{x}}{\partial t}-\frac{\partial \vec{x}}{\partial s}\left(\frac{\partial \vec{x}}{\partial t}.\frac{\partial \vec{x}}{\partial s}\right)$$\\
Donc $\frac{\partial \vec{x}}{\partial s}$ est un vecteur unité qui indique la direction tangente à la corde en chaque point. 
La métrique utilisée dans tout l'article est
$$
\eta=
\begin{bmatrix}
    -1 & 0 & 0 & 0\\
    0  & 1 & 0 & 0\\
    0  & 0 & 1 & 0\\
    0  & 0 & 0 & 1
\end{bmatrix}
$$
Par conséquent, $$\left(\frac{\partial x}{\partial \tau}\right)^2= \left(\frac{\partial \vec{x}}{\partial t}\right)^2-c^2$$et$$\left(\frac{\partial x}{\partial \sigma}\right)^2= \left(\frac{\partial \vec{x}}{\partial \sigma}\right)^2$$
Nous pouvons donc calculer que $$v_{\perp}^2=\left(\frac{\partial \vec{x}}{\partial t}\right)^2- \left(\frac{\partial \vec{x}}{\partial t}.\frac{\partial \vec{x}}{\partial s}\right)^2$$
D'autre part on a
$$\frac{ds}{d\sigma}=\vert\frac{\partial \vec{x}}{\partial \sigma}\vert$$
On obtient ainsi finalement
\begin{equation}
\frac{ds}{d\sigma}\left(1-\frac{v_{\perp}^2}{c^2}\right)^{1/2}=\frac{1}{c}\bigg[c^2\left(\frac{\partial \vec{x}}{\partial \sigma}\right)^2-\left(\frac{\partial \vec{x}}{\partial \sigma}\right)^2\left(\frac{\partial \vec{x}}{\partial t}\right)^2+\left(\frac{\partial \vec{x}}{\partial \sigma}\right)^2\left(\frac{\partial \vec{x}}{\partial t}.\frac{\partial \vec{x}}{\partial s}\right)^2\bigg]^{1/2}\\
\end{equation}
\begin{equation}
=\frac{1}{c}\bigg[-\left(\frac{\partial x}{\partial\tau}\right)^2\left(\frac{\partial x}{\partial\sigma}\right)^2+\left(\frac{\partial x}{\partial\sigma}.\frac{\partial x}{\partial\tau}\right)^2\bigg]^{1/2}
\end{equation}
Et donc on voit que
\begin{equation}
L=\frac{-1}{2\pi\alpha'\hbar c}\int_{0}^{\pi}d\sigma\frac{ds}{d\sigma}\left(1-\frac{v_{\perp}^2}{c^2}\right)^{1/2}
\end{equation}
\end{document}