\documentclass[a4paper,12pt]{article}
\def\xmu{x^\mu}
\def\vecx{\vec{x}}
\def\CP{\mathcal{P}}
\def\CL{\mathcal{L}}
\def\pt{P_\tau}
\def\vecpt{\vec{\pt}}
\usepackage[utf8]{inputenc}
\usepackage{amsmath}
\usepackage{amssymb}
\usepackage[a4paper]{geometry}
\geometry{hscale=0.80,vscale=0.80,centering}

\begin{document}


\textbf{Remarque sur la covariance des équations:}\\
Les trois premières équations de (15) sont covariantes: on peut changer de référentiel par une transformation de Lorentz et ces équations seront toujours valables. Cependant, la quatrième ne l'est pas, car elle spécifie la coordonnée $\tau$ de manière unique en fonction du "quadrivecteur" $n$. 
Si on considère seulement les trois premières équations, on peut reparamétriser:
\begin{align}
\sigma=\sigma(\tilde{\sigma},\tilde{\tau})\\
\tau=\tau(\tilde{\sigma},\tilde{\tau})
\end{align}
On a donc une liberté de paramétrisation, qui est similaire à celle qu'on a dans les équations de Maxwell (\textit{liberté de jauge}). La paramétrisation sera fixée uniquement si on spécifie les configurations initiales et finales de la corde (deux courbes \textit{type espace} sans intersection et à $\tau$ constant).
\end{document}